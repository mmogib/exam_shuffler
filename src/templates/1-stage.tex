\documentclass[dvips, 12pt,a4paper]{article}
\usepackage{latexsym, amsmath}
\usepackage{amssymb}
\topmargin=-8mm \textheight=222mm \textwidth=160mm
\oddsidemargin=7mm \evensidemargin=-5mm \thispagestyle{empty}
\newcommand{\rr}{I\hspace{-1.5mm}R}
\newcommand{\nn}{I\hspace{-1.5mm}N}
\newcommand{\cc}{C\hspace{-3mm}C}
\pagestyle{empty}


\begin{document}

Math 101
%1Page

\begin{enumerate}
    
\item If $\displaystyle f(x)=\cot\,x \cdot\,\csc\,x,$ then $\displaystyle f'\left(\frac{\pi}{4}\right)=$
\\[0.2in]
\begin{itemize}
\item [a)]$\displaystyle -3\sqrt{2}$
\item [b)]$\displaystyle 3\sqrt{2}$
\item [c)]$\displaystyle 2\sqrt{2}$
\item [d)]$\displaystyle -2\sqrt{2}$
\item [e)]$\displaystyle \sqrt{2}$
\end{itemize}


\vspace{0.8in}

\item If $\displaystyle f(x)=\ln(3x^2+x+1),$ then $f'(0)=$
\\[0.2in]
\begin{itemize}
\item [a)]1
\item [b)]0
\item [c)]2
\item [d)]3
\item [e)]-1
\end{itemize}

\newpage
%2Page


\item Consider the equation $\displaystyle (y^2-1)\,x^2+y^3=\frac{1}{x}.$ What is the value of $y'$ at $y=-1$?
\\[0.2in]
\begin{itemize}
\item [a)]$-1$
\item [b)]$-2$
\item [c)]$0$
\item [d)]$1$
\item [e)]$2$
\end{itemize}

\vspace{0.8in}

\item If $x^2+xy+y^2=1,$ then $y''=$
\\[0.2in]
\begin{itemize}
\item [a)]$\displaystyle -6/(x+2y)^3 $
\item [b)]$(2x+y)/(x+2y)^2$
\item [c)]$-3x/(x+2y)^2$
\item [d)]$2/(x+2y)^3$
\item [e)]$(x^2+y^2)/(x+2y)^3$
\end{itemize}

\newpage
%3Page


\item Find  the value of the constant $c$ such that $\displaystyle y=\frac{3}{2}\,x+6$ is tangent to $y=c\,\sqrt{x}:$
\\[0.2in]
\begin{itemize}
\item [a)]$\displaystyle 6$
\item [b)]$\displaystyle 4$
\item [c)]$\displaystyle -4$
\item [d)]$\displaystyle -6$
\item [e)]$\displaystyle 2$
\end{itemize}

\vspace{0.8in}

\item $\displaystyle \lim_{x\rightarrow\infty} \,x\,\sin\left(\frac{4}{x}\right)=$
\\[0.2in]
\begin{itemize}
\item [a)]$\displaystyle 4$
\item [b)]$\displaystyle 1$
\item [c)]$\displaystyle 1/4$
\item [d)]$\displaystyle 0$
\item [e)]$\displaystyle \mbox{Does not exist}$
\end{itemize}

\newpage
%4Page


\item If $\displaystyle h(x)=\frac{x\,\sin\,x}{g(x)};$ with $g\left(\displaystyle\frac{\pi}{2}\right)=1$ and $h'\left(\displaystyle\frac{\pi}{2}\right)=-2,$ then $g'\left(\displaystyle \frac{\pi}{2}\right)$ is :
\\[0.2in]
\begin{itemize}
\item [a)]$6/\pi$
\item [b)]$-6/\pi$
\item [c)]$0$
\item [d)]$4/\pi$
\item [e)]$-4/\pi$
\end{itemize}

\vspace{0.8in}

\item The point(s) $(x,y)$ on $\displaystyle y=\frac{x}{x-5}$ at which the tangent line(s) is (are) perpendicular to $y=5x-4$ is (are):
\\[0.2in]
\begin{itemize}
\item [a)]$(0,0) \,\mbox{and}\,(10,2)$
\item [b)]$(0,0)\,\mbox{only}$
\item [c)]$(5,2)\,\mbox{only}$
\item [d)]$(10,2)\,\mbox{only}$
\item [e)]$(0,0) \,\mbox{and}\,(5,2)$
\end{itemize}


\newpage
%5Page

\item An equation of the tangent line to the curve $y=\sqrt[3]{x}-x^3$ at $x=1$ is:
\\[0.2in]
\begin{itemize}
\item [a)]$8x+3y=8$
\item [b)]$3x+8y=3$
\item [c)]$8x-3y=8$
\item [d)]$3x-8y=3$
\item [e)]$-3x+8y=3$
\end{itemize}

\vspace{0.8in}

\item A particle moves according to the position function $s(t)=t^2\,e^{-t};\,t\geq 0,$ where $t$ is in seconds and $s$ is in meters. What is the total distance traveled by the particle during the first $6$ seconds?
\\[0.2in]
\begin{itemize}
\item [a)]$8\,e^{-2}-36\,e^{-6}$
\item [b)]$4\,e^{-2}-36\,e^{-6}$
\item [c)]$4\,e^{-2}$
\item [d)]$36\,e^{-6}$
\item [e)]$8\,e^{-2}$
\end{itemize}


\newpage
%6Page


\item If $F(x)=3f(4e^{x})\cdot\cos\,x,$ and $\displaystyle f'(4)=\frac{-1}{2}$ then $F'(0)=?$
\\[0.2in]
\begin{itemize}
\item [a)]$-6$
\item [b)]$-3$
\item [c)]$3$
\item [d)]$6$
\item [e)]$4$
\end{itemize}

\vspace{0.8in}

\item  If $\displaystyle y=\tan^{-1} \left(\frac{b+a\cos\,x}{a-b\cos\,x}\right),$ where $a$ and $b$ are non-zero constants, then $\displaystyle \frac{dy}{dx}=$
\\[0.2in]
\begin{itemize}
\item [a)]$\displaystyle \frac{-\sin\,x}{1+\cos^{2}\,x}$
\item [b)]$\displaystyle \frac{ab\,\sin\,x}{(a-b\,\cos\,x)^2}$
\item [c)]$\displaystyle \frac{a^2\,\cos\,x}{b^2 (1+\sin^{2}\,x)}$
\item [d)]$\displaystyle \frac{a^2-b^2}{(a-b\,\cos\,x)^2}$
\item [e)]$\displaystyle 0$
\end{itemize}



\newpage
%7Page


\item If the curves $y=3-x^2$ and $y=Ax^3+B$ intersect at $(1,2)$ and their tangent lines at that point are perpendicular, then $7A+B=$
\\[0.2in]
\begin{itemize}
\item [a)]$3$
\item [b)]$1$
\item [c)]$2$
\item [d)]$0$
\item [e)]$4$
\end{itemize}

\vspace{0.8in}

\item The function $y=f(x)$ satisfies the equation $xy''+y'+xy=0,$ for all $x$. If $f(0)=1,$ then $f''(0)=$
\\[0.2in]
\begin{itemize}
\item [a)]$\displaystyle \frac{-1}{2}$
\item [b)]$\displaystyle \frac{-3}{4}$
\item [c)]$3$
\item [d)]$2$
\item [e)]$\displaystyle \frac{1}{2}$
\end{itemize}

\newpage
%8Page


\item $\displaystyle \lim_{\theta\rightarrow 0} \frac{\sin(2\theta)-\cos(5\,\theta)+\tan(3\,\theta)+1}{\sin(3\,\theta)+\cos(6\,\theta)-\tan(2\theta)-1}=$
\\[0.2in]
\begin{itemize}
\item [a)]$5$
\item [b)]$4$
\item [c)]$3$
\item [d)]$2$
\item [e)]$\mbox{Does not exist}$
\end{itemize}

\vspace{0.8in}

\item If $f(0)=1$ and $f'(0)=2,$ then $(f^{-1})'(1)=$
\\[0.2in]
\begin{itemize}
\item [a)]$\displaystyle \frac{1}{2}$
\item [b)]$\displaystyle \frac{-1}{2}$
\item [c)]$\displaystyle \frac{1}{4}$
\item [d)]$\displaystyle \frac{-1}{4}$
\item [e)]$-1$
\end{itemize}


\newpage
%9Page


\item If $\displaystyle f(x)=xe^{-x},$ then $f^{(n)}(0)= $\\[0.in] $(f^{(n)}(x): n^{th}\,\mbox{derivative of $f$ at}\, x)$
\\[0.2in]
\begin{itemize}
\item [a)]$(-1)^{n+1}\cdot n$
\item [b)]$(-1)^{n}\cdot n$
\item [c)]$(-1)^{n}\cdot (n+1)$
\item [d)]$(-1)^{n+1}\cdot (n+1)$
\item [e)]$(-1)^{n}\cdot (2n)$
\end{itemize}

\vspace{0.8in}

\item The equation of the normal line to $y=\sin(\cos\,x)$ at $\displaystyle \left(\frac{\pi}{2},0\right)$
\\[0.2in]
\begin{itemize}
\item [a)]$y=x-\displaystyle \frac{\pi}{2}$
\item [b)]$y=x+\displaystyle \frac{\pi}{2}$
\item [c)]$y=-x\displaystyle +\frac{\pi}{2}$
\item [d)]$y=-x\displaystyle -\frac{\pi}{2}$
\item [e)]$y=\displaystyle \frac{\pi}{2}\,x+1$
\end{itemize}

\newpage
%10Page


\item If $y=(2x)^{4x},$ then $y'=$
\\[0.2in]
\begin{itemize}
\item [a)]$4y(1+\ln\,(2x))$
\item [b)]$4y(x+\ln\,(2x))$
\item [c)]$(4x)\cdot(2x)^{4x-1}$
\item [d)]$8y\,\ln(2x)$
\item [e)]$4xy(1+\ln\,(4x))$
\end{itemize}

\vspace{0.8in}

\item A piece of land is shaped like a right triangle. Two people start at the right angle of the triangle at the same time, and walk at the same speed along the two different legs of the triangle. If the area of the triangle formed by the positions of the two people and their starting point (the right angle) is changing at $4\,m^2/s,$ then how fast are the two people moving when they are $5m$ from the right angle?
\\[0.2in]
\begin{itemize}
\item [a)]$0.8\,m/s$
\item [b)]$1.6\,m/s$
\item [c)]$0.4\,m/s$
\item [d)]$2.0\,m/s$
\item [e)]$2.4\,m/s$
\end{itemize}

\end{enumerate}


\end{document}
