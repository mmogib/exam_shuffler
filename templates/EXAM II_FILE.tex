\documentclass[amsfonts,bezier,leqno,fleqn,12pt,a4paper]{article}

\begin{document}

\topmargin=-.75in
\textheight=26.5cm
\footskip=.3in

\oddsidemargin=-.1cm
\textwidth=16.95cm

\arraycolsep=.15in
\labelsep=.75cm

\pagestyle{myheadings}

\def \al{\alpha}
\def \b{\beta}
\def \ba{\begin{array}}
\def \del{\delta}
\def \Del{\Delta}
\def \ds{\displaystyle}
\def \fr{\frac}
\def \hf{\hfill}
\def \hl{\hline}
\def \hrf{\hrulefill}
\def \hs1{\hspace*{4mm}}
\def \inf{\infty}
\def \lam{\lambda}
\def \lan{\langle}
\def \lb{\linebreak}
\def \l{\left}
\def \la{\leftarrow}
\def \mb{\mbox}
\def \no1{\noindent}
\def \n1{\newpage}
\def \ov{\overline}
\def \p{\put}
\def \ran{\rangle}
\def \ra{\rightarrow}
\def \r{\right}
\def \s1{\sqrt}
\def \ti{\times}
\def \tr{\triangle}
\def \ts{\textstyle}
\def \th{\theta}
\def \ul{\unitlength}
\def \un1{\underline}
\def \vs1{\vspace {4mm}}
\def \v2{\vspace {3.5cm}}
\def \z{\overline{z}}

\def \bi{\begin{itemize}}
\def \ei{\end{itemize}}
\def \bt{\begin{tabular}}
\def  \et{\end{tabular}}
\def \bp{\begin{picture}}
\def \ep{\end{picture}}
\def \bc{\begin{center}}
\def \ec{\end{center}}
\def \ea{\end{array}}
\def \ba{\begin{array}}
\def \be{\begin{equation}}
\def \ee{\end{equation}}
\def \bn{\begin{enumerate}}
\def \en{\end{enumerate}}

\renewcommand{\sc}{\vspace {0.3in} \setcounter{equation}{0}}
\renewcommand{\theequation}{\alph{equation}}
\newtheorem{question}{\mbox{}}


\thicklines
\pagestyle{myheadings}

\thispagestyle{empty}

\begin{center}
\begin{large}

King Fahd University of Petroleum and Minerals\\
Department of Mathematics and Statistics\\
\vspace*{2cm}
{\bf Math 101}  \\
{\bf Exam II}  \\
{\bf Term 171}  \\
{\bf Tuesday 28/11/2017}  \\

\vspace*{3cm}
{\bf{\Huge{\fbox{EXAM COVER}}}}\\
\vspace*{2cm}
{\bf Number of versions: 4 }  \\
{\bf Number of questions: 20 }  \\
{\bf Number of Answers: 5 per question}  \\

\end{large}
\vfill

\tiny{This exam was prepared using mcqs\\}
\tiny{For questions send an email to Dr. Ibrahim Al-Lehyani (iallehyani@kaau.edu.sa)\\}

\end{center}

\newpage


\thispagestyle{empty}

\begin{center}
\begin{large}

King Fahd University of Petroleum and Minerals\\
Department of Mathematics and Statistics\\
\vspace*{4cm}
{\bf Math 101} \\
{\bf Exam II}  \\
{\bf Term 171}  \\
{\bf Tuesday 28/11/2017}  \\
{\bf Net Time Allowed: 120 minutes}  \\
\vspace*{6cm}
{\bf {\Huge{MASTER VERSION}}}  \\

\end{large}
\end{center}

\newpage


\renewcommand{\thepage}{\noindent Math 101, Term 171, Exam II \hfill Page {\bf \arabic{page} of 10} \hfill {\bf \fbox{MASTER}}}

\setcounter{page}{1}

\begin{large}
\bn


\item %1
If $\displaystyle f(x)=\cot\,x \cdot\,\csc\,x,$ then $\displaystyle f'\left(\frac{\pi}{4}\right)=$
\sc
\be
\displaystyle -3\sqrt{2}
\ee
\be
\displaystyle 3\sqrt{2}
\ee
\be
\displaystyle 2\sqrt{2}
\ee
\be
\displaystyle -2\sqrt{2}
\ee
\be
\displaystyle \sqrt{2}
\ee
\v2



\item %2
If $\displaystyle f(x)=\ln(3x^2+x+1),$ then $f'(0)=$
\sc
\be
1
\ee
\be
0
\ee
\be
2
\ee
\be
3
\ee
\be
-1
\ee
\newpage



\item %3
Consider the equation $\displaystyle (y^2-1)\,x^2+y^3=\frac{1}{x}.$ What is the value of $y'$ at $y=-1$?
\sc
\be
-1
\ee
\be
-2
\ee
\be
0
\ee
\be
1
\ee
\be
2
\ee
\v2



\item %4
If $x^2+xy+y^2=1,$ then $y''=$
\sc
\be
\displaystyle -6/(x+2y)^3 
\ee
\be
(2x+y)/(x+2y)^2
\ee
\be
-3x/(x+2y)^2
\ee
\be
2/(x+2y)^3
\ee
\be
(x^2+y^2)/(x+2y)^3
\ee
\newpage



\item %5
Find  the value of the constant $c$ such that $\displaystyle y=\frac{3}{2}\,x+6$ is tangent to $y=c\,\sqrt{x}:$
\sc
\be
6
\ee
\be
4
\ee
\be
-4
\ee
\be
-6
\ee
\be
2
\ee
\v2



\item %6
$\displaystyle \lim_{x\rightarrow\infty} \,x\,\sin\left(\frac{4}{x}\right)=$
\sc
\be
4
\ee
\be
1
\ee
\be
1/4
\ee
\be
0
\ee
\be
\mbox{Does not exist}
\ee
\newpage



\item %7
If $\displaystyle h(x)=\frac{x\,\sin\,x}{g(x)};$ with $g\left(\displaystyle\frac{\pi}{2}\right)=1$ and $h'\left(\displaystyle\frac{\pi}{2}\right)=-2,$ then $g'\left(\displaystyle \frac{\pi}{2}\right)$ is :
\sc
\be
6/\pi
\ee
\be
-6/\pi
\ee
\be
0
\ee
\be
4/\pi
\ee
\be
-4/\pi
\ee
\v2



\item %8
The point(s) $(x,y)$ on $\displaystyle y=\frac{x}{x-5}$ at which the tangent line(s) is (are) perpendicular to $y=5x-4$ is (are):
\sc
\be
(0,0) \,\mbox{and}\,(10,2)
\ee
\be
(0,0)\,\mbox{only}
\ee
\be
(5,2)\,\mbox{only}
\ee
\be
(10,2)\,\mbox{only}
\ee
\be
(0,0) \,\mbox{and}\,(5,2)
\ee
\newpage



\item %9
An equation of the tangent line to the curve $y=\sqrt[3]{x}-x^3$ at $x=1$ is:
\sc
\be
8x+3y=8
\ee
\be
3x+8y=3
\ee
\be
8x-3y=8
\ee
\be
3x-8y=3
\ee
\be
-3x+8y=3
\ee
\v2



\item %10
A particle moves according to the position function \\$s(t)=t^2\,e^{-t};\,t\geq 0,$ where $t$ is in seconds and $s$ is in meters. What is the total distance traveled by the particle during the first $6$ seconds?
\sc
\be
8\,e^{-2}-36\,e^{-6}
\ee
\be
4\,e^{-2}-36\,e^{-6}
\ee
\be
4\,e^{-2}
\ee
\be
36\,e^{-6}
\ee
\be
8\,e^{-2}
\ee
\newpage



\item %11
If $F(x)=3f(4e^{x})\cdot\cos\,x,$ and $\displaystyle f'(4)=\frac{-1}{2}$ then $F'(0)=?$
\sc
\be
-6
\ee
\be
-3
\ee
\be
3
\ee
\be
6
\ee
\be
4
\ee
\v2



\item %12
If $\displaystyle y=\tan^{-1} \left(\frac{b+a\cos\,x}{a-b\cos\,x}\right),$ where $a$ and $b$ are non-zero constants, then $\displaystyle \frac{dy}{dx}=$
\sc
\be
\displaystyle \frac{-\sin\,x}{1+\cos^{2}\,x}
\ee
\be
\displaystyle \frac{ab\,\sin\,x}{(a-b\,\cos\,x)^2}
\ee
\be
\displaystyle \frac{a^2\,\cos\,x}{b^2 (1+\sin^{2}\,x)}
\ee
\be
\displaystyle \frac{a^2-b^2}{(a-b\,\cos\,x)^2}
\ee
\be
\displaystyle 0
\ee
\newpage



\item %13
If the curves $y=3-x^2$ and $y=Ax^3+B$ intersect at $(1,2)$ and their tangent lines at that point are perpendicular, then $7A+B=$
\sc
\be
3
\ee
\be
1
\ee
\be
2
\ee
\be
0
\ee
\be
4
\ee
\v2



\item %14
The function $y=f(x)$ satisfies the equation\\ $xy''+y'+xy=0,$ for all $x$. If $f(0)=1,$ then $f''(0)=$
\sc
\be
\displaystyle \frac{-1}{2}
\ee
\be
\displaystyle \frac{-3}{4}
\ee
\be
3
\ee
\be
2
\ee
\be
\displaystyle \frac{1}{2}
\ee
\newpage



\item %15
$\displaystyle \lim_{\theta\rightarrow 0} \frac{\sin(2\theta)-\cos(5\,\theta)+\tan(3\,\theta)+1}{\sin(3\,\theta)+\cos(6\,\theta)-\tan(2\theta)-1}=$
\sc
\be
5
\ee
\be
4
\ee
\be
3
\ee
\be
2
\ee
\be
\mbox{Does not exist}
\ee
\v2



\item %16
If $f(0)=1$ and $f'(0)=2,$ then $(f^{-1})'(1)=$
\sc
\be
\displaystyle \frac{1}{2}
\ee
\be
\displaystyle \frac{-1}{2}
\ee
\be
\displaystyle \frac{1}{4}
\ee
\be
\displaystyle \frac{-1}{4}
\ee
\be
-1
\ee
\newpage



\item %17
If $\displaystyle f(x)=xe^{-x},$ then $f^{(n)}(0)= $\\[0.in] $(f^{(n)}(x): n^{th}\,\mbox{derivative of $f$ at}\, x)$
\sc
\be
(-1)^{n+1}\cdot n
\ee
\be
(-1)^{n}\cdot n
\ee
\be
(-1)^{n}\cdot (n+1)
\ee
\be
(-1)^{n+1}\cdot (n+1)
\ee
\be
(-1)^{n}\cdot (2n)
\ee
\v2



\item %18
The equation of the normal line to $y=\sin(\cos\,x)$ at $\displaystyle \left(\frac{\pi}{2},0\right)$
\sc
\be
y=x-\displaystyle \frac{\pi}{2}
\ee
\be
y=x+\displaystyle \frac{\pi}{2}
\ee
\be
y=-x\displaystyle +\frac{\pi}{2}
\ee
\be
y=-x\displaystyle -\frac{\pi}{2}
\ee
\be
y=\displaystyle \frac{\pi}{2}\,x+1
\ee
\newpage



\item %19
If $y=(2x)^{4x},$ then $y'=$
\sc
\be
4y(1+\ln\,(2x))
\ee
\be
4y(x+\ln\,(2x))
\ee
\be
(4x)\cdot(2x)^{4x-1}
\ee
\be
8y\,\ln(2x)
\ee
\be
4xy(1+\ln\,(4x))
\ee
\v2



\item %20
A piece of land is shaped like a right triangle. Two people start at the right angle of the triangle at the same time, and walk at the same speed along the two different legs of the triangle. If the area of the triangle formed by the positions of the two people and their starting point (the right angle) is changing at $4\,m^2/s,$ then how fast are the two people moving when they are $5m$ from the right angle?
\sc
\be
0.8\,m/s
\ee
\be
1.6\,m/s
\ee
\be
0.4\,m/s
\ee
\be
2.0\,m/s
\ee
\be
2.4\,m/s
\ee
\newpage



\en

\end{large}

\newpage


\thispagestyle{empty}

\begin{center}
\begin{large}

King Fahd University of Petroleum and Minerals\\
Department of Mathematics and Statistics\\
\vspace*{0.5cm}
{\bf \fbox{CODE {\small{001}}}} \hfill {\bf Math 101} \hfill {\bf \fbox{CODE {\small{001}}}} \\
{\bf Exam II}  \\
{\bf Term 171}  \\
{\bf Tuesday 28/11/2017}  \\
{\bf Net Time Allowed: 120 minutes}  \\
\vspace*{0.2cm}

\end{large}
\end{center}

\large{Name:  }\hrulefill

\vspace{3mm}

\large{ID: } \hrulefill \large{  Sec: } \hrulefill \large{.

\vspace{1cm}

\large{\bf{Check that this exam has {\underline{20}} questions.}}

\vspace{1cm}

\underline{{\large{\bf Important Instructions:}}}

\begin{enumerate}
\begin{normalsize}
\item  All types of calculators, pagers or mobile phones are NOT allowed during the examination.
\item  Use HB 2.5 pencils only.
\item  Use a good eraser. DO NOT use the erasers attached to the pencil.
\item  Write your name, ID number and Section number on the examination paper and in the upper left corner of the answer sheet.
\item  When bubbling your ID number and Section number, be sure that the bubbles match with the numbers that you write.
\item  The Test Code Number is already bubbled in your answer sheet. Make sure that it is the same as that printed on your question paper.
\item  When bubbling, make sure that the bubbled space is fully covered.
\item  When erasing a bubble, make sure that you do not leave any trace of penciling.
\end{normalsize}
\end{enumerate}

\newpage


\renewcommand{\thepage}{\noindent 
Math 101, Term 171, Exam II \hfill Page {\bf \arabic{page} of 10} \hfill {\bf \fbox{001}}}

\setcounter{page}{1}

\begin{large}

\bn


\item %1
If $x^2+xy+y^2=1,$ then $y''=$
\sc
\be
2/(x+2y)^3
\ee
\be
(2x+y)/(x+2y)^2
\ee
\be
-3x/(x+2y)^2
\ee
\be
\displaystyle -6/(x+2y)^3 
\ee
\be
(x^2+y^2)/(x+2y)^3
\ee
\v2



\item %2
Find  the value of the constant $c$ such that $\displaystyle y=\frac{3}{2}\,x+6$ is tangent to $y=c\,\sqrt{x}:$
\sc
\be
-4
\ee
\be
-6
\ee
\be
4
\ee
\be
6
\ee
\be
2
\ee

\newpage



\item %3
Consider the equation $\displaystyle (y^2-1)\,x^2+y^3=\frac{1}{x}.$ What is the value of $y'$ at $y=-1$?
\sc
\be
-2
\ee
\be
1
\ee
\be
0
\ee
\be
2
\ee
\be
-1
\ee
\v2



\item %4
If $\displaystyle f(x)=\ln(3x^2+x+1),$ then $f'(0)=$
\sc
\be
3
\ee
\be
0
\ee
\be
1
\ee
\be
2
\ee
\be
-1
\ee

\newpage



\item %5
If $\displaystyle f(x)=\cot\,x \cdot\,\csc\,x,$ then $\displaystyle f'\left(\frac{\pi}{4}\right)=$
\sc
\be
\displaystyle \sqrt{2}
\ee
\be
\displaystyle -3\sqrt{2}
\ee
\be
\displaystyle 3\sqrt{2}
\ee
\be
\displaystyle 2\sqrt{2}
\ee
\be
\displaystyle -2\sqrt{2}
\ee
\v2



\item %6
An equation of the tangent line to the curve $y=\sqrt[3]{x}-x^3$ at $x=1$ is:
\sc
\be
-3x+8y=3
\ee
\be
3x-8y=3
\ee
\be
8x+3y=8
\ee
\be
8x-3y=8
\ee
\be
3x+8y=3
\ee

\newpage



\item %7
A particle moves according to the position function \\$s(t)=t^2\,e^{-t};\,t\geq 0,$ where $t$ is in seconds and $s$ is in meters. What is the total distance traveled by the particle during the first $6$ seconds?
\sc
\be
8\,e^{-2}-36\,e^{-6}
\ee
\be
36\,e^{-6}
\ee
\be
8\,e^{-2}
\ee
\be
4\,e^{-2}-36\,e^{-6}
\ee
\be
4\,e^{-2}
\ee
\v2



\item %8
$\displaystyle \lim_{x\rightarrow\infty} \,x\,\sin\left(\frac{4}{x}\right)=$
\sc
\be
1/4
\ee
\be
4
\ee
\be
0
\ee
\be
\mbox{Does not exist}
\ee
\be
1
\ee

\newpage



\item %9
The point(s) $(x,y)$ on $\displaystyle y=\frac{x}{x-5}$ at which the tangent line(s) is (are) perpendicular to $y=5x-4$ is (are):
\sc
\be
(10,2)\,\mbox{only}
\ee
\be
(5,2)\,\mbox{only}
\ee
\be
(0,0)\,\mbox{only}
\ee
\be
(0,0) \,\mbox{and}\,(10,2)
\ee
\be
(0,0) \,\mbox{and}\,(5,2)
\ee
\v2



\item %10
If $\displaystyle h(x)=\frac{x\,\sin\,x}{g(x)};$ with $g\left(\displaystyle\frac{\pi}{2}\right)=1$ and $h'\left(\displaystyle\frac{\pi}{2}\right)=-2,$ then $g'\left(\displaystyle \frac{\pi}{2}\right)$ is :
\sc
\be
0
\ee
\be
4/\pi
\ee
\be
-6/\pi
\ee
\be
-4/\pi
\ee
\be
6/\pi
\ee

\newpage



\item %11
$\displaystyle \lim_{\theta\rightarrow 0} \frac{\sin(2\theta)-\cos(5\,\theta)+\tan(3\,\theta)+1}{\sin(3\,\theta)+\cos(6\,\theta)-\tan(2\theta)-1}=$
\sc
\be
2
\ee
\be
3
\ee
\be
\mbox{Does not exist}
\ee
\be
5
\ee
\be
4
\ee
\v2



\item %12
If $\displaystyle y=\tan^{-1} \left(\frac{b+a\cos\,x}{a-b\cos\,x}\right),$ where $a$ and $b$ are non-zero constants, then $\displaystyle \frac{dy}{dx}=$
\sc
\be
\displaystyle \frac{a^2\,\cos\,x}{b^2 (1+\sin^{2}\,x)}
\ee
\be
\displaystyle \frac{a^2-b^2}{(a-b\,\cos\,x)^2}
\ee
\be
\displaystyle \frac{ab\,\sin\,x}{(a-b\,\cos\,x)^2}
\ee
\be
\displaystyle \frac{-\sin\,x}{1+\cos^{2}\,x}
\ee
\be
\displaystyle 0
\ee

\newpage



\item %13
The function $y=f(x)$ satisfies the equation\\ $xy''+y'+xy=0,$ for all $x$. If $f(0)=1,$ then $f''(0)=$
\sc
\be
\displaystyle \frac{-1}{2}
\ee
\be
\displaystyle \frac{1}{2}
\ee
\be
3
\ee
\be
2
\ee
\be
\displaystyle \frac{-3}{4}
\ee
\v2



\item %14
If $F(x)=3f(4e^{x})\cdot\cos\,x,$ and $\displaystyle f'(4)=\frac{-1}{2}$ then $F'(0)=?$
\sc
\be
-6
\ee
\be
3
\ee
\be
6
\ee
\be
4
\ee
\be
-3
\ee

\newpage



\item %15
If the curves $y=3-x^2$ and $y=Ax^3+B$ intersect at $(1,2)$ and their tangent lines at that point are perpendicular, then $7A+B=$
\sc
\be
1
\ee
\be
0
\ee
\be
3
\ee
\be
2
\ee
\be
4
\ee
\v2



\item %16
If $\displaystyle f(x)=xe^{-x},$ then $f^{(n)}(0)= $\\[0.in] $(f^{(n)}(x): n^{th}\,\mbox{derivative of $f$ at}\, x)$
\sc
\be
(-1)^{n}\cdot n
\ee
\be
(-1)^{n}\cdot (2n)
\ee
\be
(-1)^{n+1}\cdot n
\ee
\be
(-1)^{n}\cdot (n+1)
\ee
\be
(-1)^{n+1}\cdot (n+1)
\ee

\newpage



\item %17
If $y=(2x)^{4x},$ then $y'=$
\sc
\be
4y(x+\ln\,(2x))
\ee
\be
4y(1+\ln\,(2x))
\ee
\be
8y\,\ln(2x)
\ee
\be
4xy(1+\ln\,(4x))
\ee
\be
(4x)\cdot(2x)^{4x-1}
\ee
\v2



\item %18
The equation of the normal line to $y=\sin(\cos\,x)$ at $\displaystyle \left(\frac{\pi}{2},0\right)$
\sc
\be
y=-x\displaystyle +\frac{\pi}{2}
\ee
\be
y=-x\displaystyle -\frac{\pi}{2}
\ee
\be
y=\displaystyle \frac{\pi}{2}\,x+1
\ee
\be
y=x-\displaystyle \frac{\pi}{2}
\ee
\be
y=x+\displaystyle \frac{\pi}{2}
\ee

\newpage



\item %19
If $f(0)=1$ and $f'(0)=2,$ then $(f^{-1})'(1)=$
\sc
\be
\displaystyle \frac{-1}{2}
\ee
\be
\displaystyle \frac{-1}{4}
\ee
\be
\displaystyle \frac{1}{4}
\ee
\be
\displaystyle \frac{1}{2}
\ee
\be
-1
\ee
\v2



\item %20
A piece of land is shaped like a right triangle. Two people start at the right angle of the triangle at the same time, and walk at the same speed along the two different legs of the triangle. If the area of the triangle formed by the positions of the two people and their starting point (the right angle) is changing at $4\,m^2/s,$ then how fast are the two people moving when they are $5m$ from the right angle?
\sc
\be
1.6\,m/s
\ee
\be
0.8\,m/s
\ee
\be
2.4\,m/s
\ee
\be
2.0\,m/s
\ee
\be
0.4\,m/s
\ee

\newpage



\en
\end{large}

\newpage


\renewcommand{\thepage}{\noindent Math 101, Term 171, Exam II \hfill Answer Sheet  \hfill {\bf \fbox{001}}}

\begin{Large}


\begin{tabular}{llll}
Name & .................................................& & \\
ID &   ................................& Sec & ..........\\
\end{tabular}

\vspace{10mm}


\end{Large}
\begin{normalsize}
\begin{center}
\begin{tabular}{|c|c c c c c c|c|c|c c c c c c|c|c|c c c c c c|}
\cline{1-7}\cline{9-15}
1  & a & b & c & d & e & f & \raisebox{0ex}[0cm][0cm]{\hspace{1cm}} & 36 & a & b & c & d & e & f\\ \cline{1-7}\cline{9-15}
2 & a & b & c & d & e & f & & 37& a & b & c & d & e & f\\ \cline{1-7}\cline{9-15}
3 & a & b & c & d & e & f & & 38& a & b & c & d & e & f\\ \cline{1-7}\cline{9-15}
4 & a & b & c & d & e & f & & 39& a & b & c & d & e & f\\ \cline{1-7}\cline{9-15}
5 & a & b & c & d & e & f & & 40& a & b & c & d & e & f\\ \cline{1-7}\cline{9-15}
6 & a & b & c & d & e & f & & 41& a & b & c & d & e & f\\ \cline{1-7}\cline{9-15}
7 & a & b & c & d & e & f & & 42& a & b & c & d & e & f\\ \cline{1-7}\cline{9-15}
8 & a & b & c & d & e & f & & 43& a & b & c & d & e & f\\ \cline{1-7}\cline{9-15}
9 & a & b & c & d & e & f & & 44& a & b & c & d & e & f\\ \cline{1-7}\cline{9-15}
10 & a & b & c & d & e & f & & 45& a & b & c & d & e & f\\ \cline{1-7}\cline{9-15}
11 & a & b & c & d & e & f & & 46& a & b & c & d & e & f\\ \cline{1-7}\cline{9-15}
12 & a & b & c & d & e & f & & 47& a & b & c & d & e & f\\ \cline{1-7}\cline{9-15}
13 & a & b & c & d & e & f & & 48& a & b & c & d & e & f\\ \cline{1-7}\cline{9-15}
14 & a & b & c & d & e & f & & 49& a & b & c & d & e & f\\ \cline{1-7}\cline{9-15}
15 & a & b & c & d & e & f & & 50& a & b & c & d & e & f\\ \cline{1-7}\cline{9-15}
16 & a & b & c & d & e & f & & 51& a & b & c & d & e & f\\ \cline{1-7}\cline{9-15}
17 & a & b & c & d & e & f & & 52& a & b & c & d & e & f\\ \cline{1-7}\cline{9-15}
18 & a & b & c & d & e & f & & 53& a & b & c & d & e & f\\ \cline{1-7}\cline{9-15}
19 & a & b & c & d & e & f & & 54& a & b & c & d & e & f\\ \cline{1-7}\cline{9-15}
20 & a & b & c & d & e & f & & 55& a & b & c & d & e & f\\ \cline{1-7}\cline{9-15}
21 & a & b & c & d & e & f & & 56& a & b & c & d & e & f\\ \cline{1-7}\cline{9-15}
22 & a & b & c & d & e & f & & 57& a & b & c & d & e & f\\ \cline{1-7}\cline{9-15}
23 & a & b & c & d & e & f & & 58& a & b & c & d & e & f\\ \cline{1-7}\cline{9-15}
24 & a & b & c & d & e & f & & 59& a & b & c & d & e & f\\ \cline{1-7}\cline{9-15}
25 & a & b & c & d & e & f & & 60& a & b & c & d & e & f\\ \cline{1-7}\cline{9-15}
26 & a & b & c & d & e & f & & 61& a & b & c & d & e & f\\ \cline{1-7}\cline{9-15}
27 & a & b & c & d & e & f & & 62& a & b & c & d & e & f\\ \cline{1-7}\cline{9-15}
28 & a & b & c & d & e & f & & 63& a & b & c & d & e & f\\ \cline{1-7}\cline{9-15}
29 & a & b & c & d & e & f & & 64& a & b & c & d & e & f\\ \cline{1-7}\cline{9-15}
30 & a & b & c & d & e & f & & 65& a & b & c & d & e & f\\ \cline{1-7}\cline{9-15}
31 & a & b & c & d & e & f & & 66& a & b & c & d & e & f\\ \cline{1-7}\cline{9-15}
32 & a & b & c & d & e & f & & 67& a & b & c & d & e & f\\ \cline{1-7}\cline{9-15}
33 & a & b & c & d & e & f & & 68& a & b & c & d & e & f\\ \cline{1-7}\cline{9-15}
34 & a & b & c & d & e & f & & 69& a & b & c & d & e & f\\ \cline{1-7}\cline{9-15}
35 & a & b & c & d & e & f & & 70& a & b & c & d & e & f\\ \cline{1-7}\cline{9-15}
\end{tabular}\end{center}
\end{normalsize}
\newpage
\thispagestyle{empty}

\begin{center}
\begin{large}

King Fahd University of Petroleum and Minerals\\
Department of Mathematics and Statistics\\
\vspace*{0.5cm}
{\bf \fbox{CODE {\small{002}}}} \hfill {\bf Math 101} \hfill {\bf \fbox{CODE {\small{002}}}} \\
{\bf Exam II}  \\
{\bf Term 171}  \\
{\bf Tuesday 28/11/2017}  \\
{\bf Net Time Allowed: 120 minutes}  \\
\vspace*{0.2cm}

\end{large}
\end{center}

\large{Name:  }\hrulefill

\vspace{3mm}

\large{ID: } \hrulefill \large{  Sec: } \hrulefill \large{.

\vspace{1cm}

\large{\bf{Check that this exam has {\underline{20}} questions.}}

\vspace{1cm}

\underline{{\large{\bf Important Instructions:}}}

\begin{enumerate}
\begin{normalsize}
\item  All types of calculators, pagers or mobile phones are NOT allowed during the examination.
\item  Use HB 2.5 pencils only.
\item  Use a good eraser. DO NOT use the erasers attached to the pencil.
\item  Write your name, ID number and Section number on the examination paper and in the upper left corner of the answer sheet.
\item  When bubbling your ID number and Section number, be sure that the bubbles match with the numbers that you write.
\item  The Test Code Number is already bubbled in your answer sheet. Make sure that it is the same as that printed on your question paper.
\item  When bubbling, make sure that the bubbled space is fully covered.
\item  When erasing a bubble, make sure that you do not leave any trace of penciling.
\end{normalsize}
\end{enumerate}

\newpage


\renewcommand{\thepage}{\noindent 
Math 101, Term 171, Exam II \hfill Page {\bf \arabic{page} of 10} \hfill {\bf \fbox{002}}}

\setcounter{page}{1}

\begin{large}

\bn


\item %1
If $x^2+xy+y^2=1,$ then $y''=$
\sc
\be
-3x/(x+2y)^2
\ee
\be
(2x+y)/(x+2y)^2
\ee
\be
\displaystyle -6/(x+2y)^3 
\ee
\be
2/(x+2y)^3
\ee
\be
(x^2+y^2)/(x+2y)^3
\ee
\v2



\item %2
If $\displaystyle f(x)=\ln(3x^2+x+1),$ then $f'(0)=$
\sc
\be
2
\ee
\be
0
\ee
\be
1
\ee
\be
3
\ee
\be
-1
\ee

\newpage



\item %3
If $\displaystyle f(x)=\cot\,x \cdot\,\csc\,x,$ then $\displaystyle f'\left(\frac{\pi}{4}\right)=$
\sc
\be
\displaystyle 3\sqrt{2}
\ee
\be
\displaystyle -3\sqrt{2}
\ee
\be
\displaystyle \sqrt{2}
\ee
\be
\displaystyle 2\sqrt{2}
\ee
\be
\displaystyle -2\sqrt{2}
\ee
\v2



\item %4
Consider the equation $\displaystyle (y^2-1)\,x^2+y^3=\frac{1}{x}.$ What is the value of $y'$ at $y=-1$?
\sc
\be
-2
\ee
\be
0
\ee
\be
-1
\ee
\be
2
\ee
\be
1
\ee

\newpage



\item %5
Find  the value of the constant $c$ such that $\displaystyle y=\frac{3}{2}\,x+6$ is tangent to $y=c\,\sqrt{x}:$
\sc
\be
4
\ee
\be
2
\ee
\be
-4
\ee
\be
-6
\ee
\be
6
\ee
\v2



\item %6
If $\displaystyle h(x)=\frac{x\,\sin\,x}{g(x)};$ with $g\left(\displaystyle\frac{\pi}{2}\right)=1$ and $h'\left(\displaystyle\frac{\pi}{2}\right)=-2,$ then $g'\left(\displaystyle \frac{\pi}{2}\right)$ is :
\sc
\be
-4/\pi
\ee
\be
4/\pi
\ee
\be
0
\ee
\be
6/\pi
\ee
\be
-6/\pi
\ee

\newpage



\item %7
The point(s) $(x,y)$ on $\displaystyle y=\frac{x}{x-5}$ at which the tangent line(s) is (are) perpendicular to $y=5x-4$ is (are):
\sc
\be
(0,0) \,\mbox{and}\,(10,2)
\ee
\be
(5,2)\,\mbox{only}
\ee
\be
(10,2)\,\mbox{only}
\ee
\be
(0,0) \,\mbox{and}\,(5,2)
\ee
\be
(0,0)\,\mbox{only}
\ee
\v2



\item %8
A particle moves according to the position function \\$s(t)=t^2\,e^{-t};\,t\geq 0,$ where $t$ is in seconds and $s$ is in meters. What is the total distance traveled by the particle during the first $6$ seconds?
\sc
\be
8\,e^{-2}-36\,e^{-6}
\ee
\be
36\,e^{-6}
\ee
\be
8\,e^{-2}
\ee
\be
4\,e^{-2}
\ee
\be
4\,e^{-2}-36\,e^{-6}
\ee

\newpage



\item %9
$\displaystyle \lim_{x\rightarrow\infty} \,x\,\sin\left(\frac{4}{x}\right)=$
\sc
\be
\mbox{Does not exist}
\ee
\be
1/4
\ee
\be
4
\ee
\be
0
\ee
\be
1
\ee
\v2



\item %10
An equation of the tangent line to the curve $y=\sqrt[3]{x}-x^3$ at $x=1$ is:
\sc
\be
3x+8y=3
\ee
\be
3x-8y=3
\ee
\be
8x-3y=8
\ee
\be
-3x+8y=3
\ee
\be
8x+3y=8
\ee

\newpage



\item %11
If $\displaystyle y=\tan^{-1} \left(\frac{b+a\cos\,x}{a-b\cos\,x}\right),$ where $a$ and $b$ are non-zero constants, then $\displaystyle \frac{dy}{dx}=$
\sc
\be
\displaystyle \frac{-\sin\,x}{1+\cos^{2}\,x}
\ee
\be
\displaystyle 0
\ee
\be
\displaystyle \frac{a^2\,\cos\,x}{b^2 (1+\sin^{2}\,x)}
\ee
\be
\displaystyle \frac{a^2-b^2}{(a-b\,\cos\,x)^2}
\ee
\be
\displaystyle \frac{ab\,\sin\,x}{(a-b\,\cos\,x)^2}
\ee
\v2



\item %12
$\displaystyle \lim_{\theta\rightarrow 0} \frac{\sin(2\theta)-\cos(5\,\theta)+\tan(3\,\theta)+1}{\sin(3\,\theta)+\cos(6\,\theta)-\tan(2\theta)-1}=$
\sc
\be
3
\ee
\be
4
\ee
\be
\mbox{Does not exist}
\ee
\be
5
\ee
\be
2
\ee

\newpage



\item %13
The function $y=f(x)$ satisfies the equation\\ $xy''+y'+xy=0,$ for all $x$. If $f(0)=1,$ then $f''(0)=$
\sc
\be
2
\ee
\be
\displaystyle \frac{-1}{2}
\ee
\be
3
\ee
\be
\displaystyle \frac{1}{2}
\ee
\be
\displaystyle \frac{-3}{4}
\ee
\v2



\item %14
If $F(x)=3f(4e^{x})\cdot\cos\,x,$ and $\displaystyle f'(4)=\frac{-1}{2}$ then $F'(0)=?$
\sc
\be
6
\ee
\be
3
\ee
\be
-3
\ee
\be
4
\ee
\be
-6
\ee

\newpage



\item %15
If the curves $y=3-x^2$ and $y=Ax^3+B$ intersect at $(1,2)$ and their tangent lines at that point are perpendicular, then $7A+B=$
\sc
\be
3
\ee
\be
2
\ee
\be
0
\ee
\be
4
\ee
\be
1
\ee
\v2



\item %16
If $\displaystyle f(x)=xe^{-x},$ then $f^{(n)}(0)= $\\[0.in] $(f^{(n)}(x): n^{th}\,\mbox{derivative of $f$ at}\, x)$
\sc
\be
(-1)^{n}\cdot (n+1)
\ee
\be
(-1)^{n}\cdot (2n)
\ee
\be
(-1)^{n+1}\cdot (n+1)
\ee
\be
(-1)^{n}\cdot n
\ee
\be
(-1)^{n+1}\cdot n
\ee

\newpage



\item %17
If $f(0)=1$ and $f'(0)=2,$ then $(f^{-1})'(1)=$
\sc
\be
\displaystyle \frac{-1}{4}
\ee
\be
\displaystyle \frac{-1}{2}
\ee
\be
-1
\ee
\be
\displaystyle \frac{1}{2}
\ee
\be
\displaystyle \frac{1}{4}
\ee
\v2



\item %18
A piece of land is shaped like a right triangle. Two people start at the right angle of the triangle at the same time, and walk at the same speed along the two different legs of the triangle. If the area of the triangle formed by the positions of the two people and their starting point (the right angle) is changing at $4\,m^2/s,$ then how fast are the two people moving when they are $5m$ from the right angle?
\sc
\be
0.8\,m/s
\ee
\be
0.4\,m/s
\ee
\be
1.6\,m/s
\ee
\be
2.4\,m/s
\ee
\be
2.0\,m/s
\ee

\newpage



\item %19
The equation of the normal line to $y=\sin(\cos\,x)$ at $\displaystyle \left(\frac{\pi}{2},0\right)$
\sc
\be
y=x-\displaystyle \frac{\pi}{2}
\ee
\be
y=-x\displaystyle +\frac{\pi}{2}
\ee
\be
y=x+\displaystyle \frac{\pi}{2}
\ee
\be
y=\displaystyle \frac{\pi}{2}\,x+1
\ee
\be
y=-x\displaystyle -\frac{\pi}{2}
\ee
\v2



\item %20
If $y=(2x)^{4x},$ then $y'=$
\sc
\be
(4x)\cdot(2x)^{4x-1}
\ee
\be
4xy(1+\ln\,(4x))
\ee
\be
4y(1+\ln\,(2x))
\ee
\be
8y\,\ln(2x)
\ee
\be
4y(x+\ln\,(2x))
\ee

\newpage



\en
\end{large}

\newpage


\renewcommand{\thepage}{\noindent Math 101, Term 171, Exam II \hfill Answer Sheet  \hfill {\bf \fbox{002}}}

\begin{Large}


\begin{tabular}{llll}
Name & .................................................& & \\
ID &   ................................& Sec & ..........\\
\end{tabular}

\vspace{10mm}


\end{Large}
\begin{normalsize}
\begin{center}
\begin{tabular}{|c|c c c c c c|c|c|c c c c c c|c|c|c c c c c c|}
\cline{1-7}\cline{9-15}
1  & a & b & c & d & e & f & \raisebox{0ex}[0cm][0cm]{\hspace{1cm}} & 36 & a & b & c & d & e & f\\ \cline{1-7}\cline{9-15}
2 & a & b & c & d & e & f & & 37& a & b & c & d & e & f\\ \cline{1-7}\cline{9-15}
3 & a & b & c & d & e & f & & 38& a & b & c & d & e & f\\ \cline{1-7}\cline{9-15}
4 & a & b & c & d & e & f & & 39& a & b & c & d & e & f\\ \cline{1-7}\cline{9-15}
5 & a & b & c & d & e & f & & 40& a & b & c & d & e & f\\ \cline{1-7}\cline{9-15}
6 & a & b & c & d & e & f & & 41& a & b & c & d & e & f\\ \cline{1-7}\cline{9-15}
7 & a & b & c & d & e & f & & 42& a & b & c & d & e & f\\ \cline{1-7}\cline{9-15}
8 & a & b & c & d & e & f & & 43& a & b & c & d & e & f\\ \cline{1-7}\cline{9-15}
9 & a & b & c & d & e & f & & 44& a & b & c & d & e & f\\ \cline{1-7}\cline{9-15}
10 & a & b & c & d & e & f & & 45& a & b & c & d & e & f\\ \cline{1-7}\cline{9-15}
11 & a & b & c & d & e & f & & 46& a & b & c & d & e & f\\ \cline{1-7}\cline{9-15}
12 & a & b & c & d & e & f & & 47& a & b & c & d & e & f\\ \cline{1-7}\cline{9-15}
13 & a & b & c & d & e & f & & 48& a & b & c & d & e & f\\ \cline{1-7}\cline{9-15}
14 & a & b & c & d & e & f & & 49& a & b & c & d & e & f\\ \cline{1-7}\cline{9-15}
15 & a & b & c & d & e & f & & 50& a & b & c & d & e & f\\ \cline{1-7}\cline{9-15}
16 & a & b & c & d & e & f & & 51& a & b & c & d & e & f\\ \cline{1-7}\cline{9-15}
17 & a & b & c & d & e & f & & 52& a & b & c & d & e & f\\ \cline{1-7}\cline{9-15}
18 & a & b & c & d & e & f & & 53& a & b & c & d & e & f\\ \cline{1-7}\cline{9-15}
19 & a & b & c & d & e & f & & 54& a & b & c & d & e & f\\ \cline{1-7}\cline{9-15}
20 & a & b & c & d & e & f & & 55& a & b & c & d & e & f\\ \cline{1-7}\cline{9-15}
21 & a & b & c & d & e & f & & 56& a & b & c & d & e & f\\ \cline{1-7}\cline{9-15}
22 & a & b & c & d & e & f & & 57& a & b & c & d & e & f\\ \cline{1-7}\cline{9-15}
23 & a & b & c & d & e & f & & 58& a & b & c & d & e & f\\ \cline{1-7}\cline{9-15}
24 & a & b & c & d & e & f & & 59& a & b & c & d & e & f\\ \cline{1-7}\cline{9-15}
25 & a & b & c & d & e & f & & 60& a & b & c & d & e & f\\ \cline{1-7}\cline{9-15}
26 & a & b & c & d & e & f & & 61& a & b & c & d & e & f\\ \cline{1-7}\cline{9-15}
27 & a & b & c & d & e & f & & 62& a & b & c & d & e & f\\ \cline{1-7}\cline{9-15}
28 & a & b & c & d & e & f & & 63& a & b & c & d & e & f\\ \cline{1-7}\cline{9-15}
29 & a & b & c & d & e & f & & 64& a & b & c & d & e & f\\ \cline{1-7}\cline{9-15}
30 & a & b & c & d & e & f & & 65& a & b & c & d & e & f\\ \cline{1-7}\cline{9-15}
31 & a & b & c & d & e & f & & 66& a & b & c & d & e & f\\ \cline{1-7}\cline{9-15}
32 & a & b & c & d & e & f & & 67& a & b & c & d & e & f\\ \cline{1-7}\cline{9-15}
33 & a & b & c & d & e & f & & 68& a & b & c & d & e & f\\ \cline{1-7}\cline{9-15}
34 & a & b & c & d & e & f & & 69& a & b & c & d & e & f\\ \cline{1-7}\cline{9-15}
35 & a & b & c & d & e & f & & 70& a & b & c & d & e & f\\ \cline{1-7}\cline{9-15}
\end{tabular}\end{center}
\end{normalsize}
\newpage
\thispagestyle{empty}

\begin{center}
\begin{large}

King Fahd University of Petroleum and Minerals\\
Department of Mathematics and Statistics\\
\vspace*{0.5cm}
{\bf \fbox{CODE {\small{003}}}} \hfill {\bf Math 101} \hfill {\bf \fbox{CODE {\small{003}}}} \\
{\bf Exam II}  \\
{\bf Term 171}  \\
{\bf Tuesday 28/11/2017}  \\
{\bf Net Time Allowed: 120 minutes}  \\
\vspace*{0.2cm}

\end{large}
\end{center}

\large{Name:  }\hrulefill

\vspace{3mm}

\large{ID: } \hrulefill \large{  Sec: } \hrulefill \large{.

\vspace{1cm}

\large{\bf{Check that this exam has {\underline{20}} questions.}}

\vspace{1cm}

\underline{{\large{\bf Important Instructions:}}}

\begin{enumerate}
\begin{normalsize}
\item  All types of calculators, pagers or mobile phones are NOT allowed during the examination.
\item  Use HB 2.5 pencils only.
\item  Use a good eraser. DO NOT use the erasers attached to the pencil.
\item  Write your name, ID number and Section number on the examination paper and in the upper left corner of the answer sheet.
\item  When bubbling your ID number and Section number, be sure that the bubbles match with the numbers that you write.
\item  The Test Code Number is already bubbled in your answer sheet. Make sure that it is the same as that printed on your question paper.
\item  When bubbling, make sure that the bubbled space is fully covered.
\item  When erasing a bubble, make sure that you do not leave any trace of penciling.
\end{normalsize}
\end{enumerate}

\newpage


\renewcommand{\thepage}{\noindent 
Math 101, Term 171, Exam II \hfill Page {\bf \arabic{page} of 10} \hfill {\bf \fbox{003}}}

\setcounter{page}{1}

\begin{large}

\bn


\item %1
If $\displaystyle f(x)=\ln(3x^2+x+1),$ then $f'(0)=$
\sc
\be
0
\ee
\be
3
\ee
\be
1
\ee
\be
-1
\ee
\be
2
\ee
\v2



\item %2
If $x^2+xy+y^2=1,$ then $y''=$
\sc
\be
\displaystyle -6/(x+2y)^3 
\ee
\be
2/(x+2y)^3
\ee
\be
(2x+y)/(x+2y)^2
\ee
\be
-3x/(x+2y)^2
\ee
\be
(x^2+y^2)/(x+2y)^3
\ee

\newpage



\item %3
If $\displaystyle f(x)=\cot\,x \cdot\,\csc\,x,$ then $\displaystyle f'\left(\frac{\pi}{4}\right)=$
\sc
\be
\displaystyle 2\sqrt{2}
\ee
\be
\displaystyle -2\sqrt{2}
\ee
\be
\displaystyle 3\sqrt{2}
\ee
\be
\displaystyle -3\sqrt{2}
\ee
\be
\displaystyle \sqrt{2}
\ee
\v2



\item %4
Consider the equation $\displaystyle (y^2-1)\,x^2+y^3=\frac{1}{x}.$ What is the value of $y'$ at $y=-1$?
\sc
\be
-1
\ee
\be
-2
\ee
\be
2
\ee
\be
0
\ee
\be
1
\ee

\newpage



\item %5
Find  the value of the constant $c$ such that $\displaystyle y=\frac{3}{2}\,x+6$ is tangent to $y=c\,\sqrt{x}:$
\sc
\be
4
\ee
\be
2
\ee
\be
6
\ee
\be
-6
\ee
\be
-4
\ee
\v2



\item %6
If $\displaystyle h(x)=\frac{x\,\sin\,x}{g(x)};$ with $g\left(\displaystyle\frac{\pi}{2}\right)=1$ and $h'\left(\displaystyle\frac{\pi}{2}\right)=-2,$ then $g'\left(\displaystyle \frac{\pi}{2}\right)$ is :
\sc
\be
0
\ee
\be
-4/\pi
\ee
\be
4/\pi
\ee
\be
-6/\pi
\ee
\be
6/\pi
\ee

\newpage



\item %7
The point(s) $(x,y)$ on $\displaystyle y=\frac{x}{x-5}$ at which the tangent line(s) is (are) perpendicular to $y=5x-4$ is (are):
\sc
\be
(5,2)\,\mbox{only}
\ee
\be
(0,0)\,\mbox{only}
\ee
\be
(0,0) \,\mbox{and}\,(10,2)
\ee
\be
(10,2)\,\mbox{only}
\ee
\be
(0,0) \,\mbox{and}\,(5,2)
\ee
\v2



\item %8
A particle moves according to the position function \\$s(t)=t^2\,e^{-t};\,t\geq 0,$ where $t$ is in seconds and $s$ is in meters. What is the total distance traveled by the particle during the first $6$ seconds?
\sc
\be
8\,e^{-2}
\ee
\be
4\,e^{-2}-36\,e^{-6}
\ee
\be
4\,e^{-2}
\ee
\be
8\,e^{-2}-36\,e^{-6}
\ee
\be
36\,e^{-6}
\ee

\newpage



\item %9
An equation of the tangent line to the curve $y=\sqrt[3]{x}-x^3$ at $x=1$ is:
\sc
\be
3x-8y=3
\ee
\be
-3x+8y=3
\ee
\be
8x+3y=8
\ee
\be
8x-3y=8
\ee
\be
3x+8y=3
\ee
\v2



\item %10
$\displaystyle \lim_{x\rightarrow\infty} \,x\,\sin\left(\frac{4}{x}\right)=$
\sc
\be
1
\ee
\be
1/4
\ee
\be
\mbox{Does not exist}
\ee
\be
0
\ee
\be
4
\ee

\newpage



\item %11
If the curves $y=3-x^2$ and $y=Ax^3+B$ intersect at $(1,2)$ and their tangent lines at that point are perpendicular, then $7A+B=$
\sc
\be
1
\ee
\be
3
\ee
\be
0
\ee
\be
2
\ee
\be
4
\ee
\v2



\item %12
If $\displaystyle y=\tan^{-1} \left(\frac{b+a\cos\,x}{a-b\cos\,x}\right),$ where $a$ and $b$ are non-zero constants, then $\displaystyle \frac{dy}{dx}=$
\sc
\be
\displaystyle \frac{a^2\,\cos\,x}{b^2 (1+\sin^{2}\,x)}
\ee
\be
\displaystyle \frac{a^2-b^2}{(a-b\,\cos\,x)^2}
\ee
\be
\displaystyle \frac{-\sin\,x}{1+\cos^{2}\,x}
\ee
\be
\displaystyle 0
\ee
\be
\displaystyle \frac{ab\,\sin\,x}{(a-b\,\cos\,x)^2}
\ee

\newpage



\item %13
If $F(x)=3f(4e^{x})\cdot\cos\,x,$ and $\displaystyle f'(4)=\frac{-1}{2}$ then $F'(0)=?$
\sc
\be
-3
\ee
\be
-6
\ee
\be
3
\ee
\be
4
\ee
\be
6
\ee
\v2



\item %14
$\displaystyle \lim_{\theta\rightarrow 0} \frac{\sin(2\theta)-\cos(5\,\theta)+\tan(3\,\theta)+1}{\sin(3\,\theta)+\cos(6\,\theta)-\tan(2\theta)-1}=$
\sc
\be
2
\ee
\be
3
\ee
\be
\mbox{Does not exist}
\ee
\be
4
\ee
\be
5
\ee

\newpage



\item %15
The function $y=f(x)$ satisfies the equation\\ $xy''+y'+xy=0,$ for all $x$. If $f(0)=1,$ then $f''(0)=$
\sc
\be
\displaystyle \frac{1}{2}
\ee
\be
3
\ee
\be
\displaystyle \frac{-1}{2}
\ee
\be
2
\ee
\be
\displaystyle \frac{-3}{4}
\ee
\v2



\item %16
If $y=(2x)^{4x},$ then $y'=$
\sc
\be
4xy(1+\ln\,(4x))
\ee
\be
8y\,\ln(2x)
\ee
\be
4y(1+\ln\,(2x))
\ee
\be
(4x)\cdot(2x)^{4x-1}
\ee
\be
4y(x+\ln\,(2x))
\ee

\newpage



\item %17
The equation of the normal line to $y=\sin(\cos\,x)$ at $\displaystyle \left(\frac{\pi}{2},0\right)$
\sc
\be
y=-x\displaystyle -\frac{\pi}{2}
\ee
\be
y=x-\displaystyle \frac{\pi}{2}
\ee
\be
y=\displaystyle \frac{\pi}{2}\,x+1
\ee
\be
y=x+\displaystyle \frac{\pi}{2}
\ee
\be
y=-x\displaystyle +\frac{\pi}{2}
\ee
\v2



\item %18
A piece of land is shaped like a right triangle. Two people start at the right angle of the triangle at the same time, and walk at the same speed along the two different legs of the triangle. If the area of the triangle formed by the positions of the two people and their starting point (the right angle) is changing at $4\,m^2/s,$ then how fast are the two people moving when they are $5m$ from the right angle?
\sc
\be
0.8\,m/s
\ee
\be
0.4\,m/s
\ee
\be
1.6\,m/s
\ee
\be
2.0\,m/s
\ee
\be
2.4\,m/s
\ee

\newpage



\item %19
If $f(0)=1$ and $f'(0)=2,$ then $(f^{-1})'(1)=$
\sc
\be
-1
\ee
\be
\displaystyle \frac{-1}{4}
\ee
\be
\displaystyle \frac{1}{4}
\ee
\be
\displaystyle \frac{1}{2}
\ee
\be
\displaystyle \frac{-1}{2}
\ee
\v2



\item %20
If $\displaystyle f(x)=xe^{-x},$ then $f^{(n)}(0)= $\\[0.in] $(f^{(n)}(x): n^{th}\,\mbox{derivative of $f$ at}\, x)$
\sc
\be
(-1)^{n+1}\cdot n
\ee
\be
(-1)^{n+1}\cdot (n+1)
\ee
\be
(-1)^{n}\cdot (2n)
\ee
\be
(-1)^{n}\cdot (n+1)
\ee
\be
(-1)^{n}\cdot n
\ee

\newpage



\en
\end{large}

\newpage


\renewcommand{\thepage}{\noindent Math 101, Term 171, Exam II \hfill Answer Sheet  \hfill {\bf \fbox{003}}}

\begin{Large}


\begin{tabular}{llll}
Name & .................................................& & \\
ID &   ................................& Sec & ..........\\
\end{tabular}

\vspace{10mm}


\end{Large}
\begin{normalsize}
\begin{center}
\begin{tabular}{|c|c c c c c c|c|c|c c c c c c|c|c|c c c c c c|}
\cline{1-7}\cline{9-15}
1  & a & b & c & d & e & f & \raisebox{0ex}[0cm][0cm]{\hspace{1cm}} & 36 & a & b & c & d & e & f\\ \cline{1-7}\cline{9-15}
2 & a & b & c & d & e & f & & 37& a & b & c & d & e & f\\ \cline{1-7}\cline{9-15}
3 & a & b & c & d & e & f & & 38& a & b & c & d & e & f\\ \cline{1-7}\cline{9-15}
4 & a & b & c & d & e & f & & 39& a & b & c & d & e & f\\ \cline{1-7}\cline{9-15}
5 & a & b & c & d & e & f & & 40& a & b & c & d & e & f\\ \cline{1-7}\cline{9-15}
6 & a & b & c & d & e & f & & 41& a & b & c & d & e & f\\ \cline{1-7}\cline{9-15}
7 & a & b & c & d & e & f & & 42& a & b & c & d & e & f\\ \cline{1-7}\cline{9-15}
8 & a & b & c & d & e & f & & 43& a & b & c & d & e & f\\ \cline{1-7}\cline{9-15}
9 & a & b & c & d & e & f & & 44& a & b & c & d & e & f\\ \cline{1-7}\cline{9-15}
10 & a & b & c & d & e & f & & 45& a & b & c & d & e & f\\ \cline{1-7}\cline{9-15}
11 & a & b & c & d & e & f & & 46& a & b & c & d & e & f\\ \cline{1-7}\cline{9-15}
12 & a & b & c & d & e & f & & 47& a & b & c & d & e & f\\ \cline{1-7}\cline{9-15}
13 & a & b & c & d & e & f & & 48& a & b & c & d & e & f\\ \cline{1-7}\cline{9-15}
14 & a & b & c & d & e & f & & 49& a & b & c & d & e & f\\ \cline{1-7}\cline{9-15}
15 & a & b & c & d & e & f & & 50& a & b & c & d & e & f\\ \cline{1-7}\cline{9-15}
16 & a & b & c & d & e & f & & 51& a & b & c & d & e & f\\ \cline{1-7}\cline{9-15}
17 & a & b & c & d & e & f & & 52& a & b & c & d & e & f\\ \cline{1-7}\cline{9-15}
18 & a & b & c & d & e & f & & 53& a & b & c & d & e & f\\ \cline{1-7}\cline{9-15}
19 & a & b & c & d & e & f & & 54& a & b & c & d & e & f\\ \cline{1-7}\cline{9-15}
20 & a & b & c & d & e & f & & 55& a & b & c & d & e & f\\ \cline{1-7}\cline{9-15}
21 & a & b & c & d & e & f & & 56& a & b & c & d & e & f\\ \cline{1-7}\cline{9-15}
22 & a & b & c & d & e & f & & 57& a & b & c & d & e & f\\ \cline{1-7}\cline{9-15}
23 & a & b & c & d & e & f & & 58& a & b & c & d & e & f\\ \cline{1-7}\cline{9-15}
24 & a & b & c & d & e & f & & 59& a & b & c & d & e & f\\ \cline{1-7}\cline{9-15}
25 & a & b & c & d & e & f & & 60& a & b & c & d & e & f\\ \cline{1-7}\cline{9-15}
26 & a & b & c & d & e & f & & 61& a & b & c & d & e & f\\ \cline{1-7}\cline{9-15}
27 & a & b & c & d & e & f & & 62& a & b & c & d & e & f\\ \cline{1-7}\cline{9-15}
28 & a & b & c & d & e & f & & 63& a & b & c & d & e & f\\ \cline{1-7}\cline{9-15}
29 & a & b & c & d & e & f & & 64& a & b & c & d & e & f\\ \cline{1-7}\cline{9-15}
30 & a & b & c & d & e & f & & 65& a & b & c & d & e & f\\ \cline{1-7}\cline{9-15}
31 & a & b & c & d & e & f & & 66& a & b & c & d & e & f\\ \cline{1-7}\cline{9-15}
32 & a & b & c & d & e & f & & 67& a & b & c & d & e & f\\ \cline{1-7}\cline{9-15}
33 & a & b & c & d & e & f & & 68& a & b & c & d & e & f\\ \cline{1-7}\cline{9-15}
34 & a & b & c & d & e & f & & 69& a & b & c & d & e & f\\ \cline{1-7}\cline{9-15}
35 & a & b & c & d & e & f & & 70& a & b & c & d & e & f\\ \cline{1-7}\cline{9-15}
\end{tabular}\end{center}
\end{normalsize}
\newpage
\thispagestyle{empty}

\begin{center}
\begin{large}

King Fahd University of Petroleum and Minerals\\
Department of Mathematics and Statistics\\
\vspace*{0.5cm}
{\bf \fbox{CODE {\small{004}}}} \hfill {\bf Math 101} \hfill {\bf \fbox{CODE {\small{004}}}} \\
{\bf Exam II}  \\
{\bf Term 171}  \\
{\bf Tuesday 28/11/2017}  \\
{\bf Net Time Allowed: 120 minutes}  \\
\vspace*{0.2cm}

\end{large}
\end{center}

\large{Name:  }\hrulefill

\vspace{3mm}

\large{ID: } \hrulefill \large{  Sec: } \hrulefill \large{.

\vspace{1cm}

\large{\bf{Check that this exam has {\underline{20}} questions.}}

\vspace{1cm}

\underline{{\large{\bf Important Instructions:}}}

\begin{enumerate}
\begin{normalsize}
\item  All types of calculators, pagers or mobile phones are NOT allowed during the examination.
\item  Use HB 2.5 pencils only.
\item  Use a good eraser. DO NOT use the erasers attached to the pencil.
\item  Write your name, ID number and Section number on the examination paper and in the upper left corner of the answer sheet.
\item  When bubbling your ID number and Section number, be sure that the bubbles match with the numbers that you write.
\item  The Test Code Number is already bubbled in your answer sheet. Make sure that it is the same as that printed on your question paper.
\item  When bubbling, make sure that the bubbled space is fully covered.
\item  When erasing a bubble, make sure that you do not leave any trace of penciling.
\end{normalsize}
\end{enumerate}

\newpage


\renewcommand{\thepage}{\noindent 
Math 101, Term 171, Exam II \hfill Page {\bf \arabic{page} of 10} \hfill {\bf \fbox{004}}}

\setcounter{page}{1}

\begin{large}

\bn


\item %1
Consider the equation $\displaystyle (y^2-1)\,x^2+y^3=\frac{1}{x}.$ What is the value of $y'$ at $y=-1$?
\sc
\be
2
\ee
\be
-2
\ee
\be
0
\ee
\be
1
\ee
\be
-1
\ee
\v2



\item %2
If $\displaystyle f(x)=\cot\,x \cdot\,\csc\,x,$ then $\displaystyle f'\left(\frac{\pi}{4}\right)=$
\sc
\be
\displaystyle 3\sqrt{2}
\ee
\be
\displaystyle \sqrt{2}
\ee
\be
\displaystyle 2\sqrt{2}
\ee
\be
\displaystyle -2\sqrt{2}
\ee
\be
\displaystyle -3\sqrt{2}
\ee

\newpage



\item %3
Find  the value of the constant $c$ such that $\displaystyle y=\frac{3}{2}\,x+6$ is tangent to $y=c\,\sqrt{x}:$
\sc
\be
6
\ee
\be
-4
\ee
\be
2
\ee
\be
4
\ee
\be
-6
\ee
\v2



\item %4
If $x^2+xy+y^2=1,$ then $y''=$
\sc
\be
\displaystyle -6/(x+2y)^3 
\ee
\be
(x^2+y^2)/(x+2y)^3
\ee
\be
2/(x+2y)^3
\ee
\be
(2x+y)/(x+2y)^2
\ee
\be
-3x/(x+2y)^2
\ee

\newpage



\item %5
If $\displaystyle f(x)=\ln(3x^2+x+1),$ then $f'(0)=$
\sc
\be
1
\ee
\be
2
\ee
\be
3
\ee
\be
0
\ee
\be
-1
\ee
\v2



\item %6
An equation of the tangent line to the curve $y=\sqrt[3]{x}-x^3$ at $x=1$ is:
\sc
\be
8x-3y=8
\ee
\be
3x-8y=3
\ee
\be
3x+8y=3
\ee
\be
8x+3y=8
\ee
\be
-3x+8y=3
\ee

\newpage



\item %7
The point(s) $(x,y)$ on $\displaystyle y=\frac{x}{x-5}$ at which the tangent line(s) is (are) perpendicular to $y=5x-4$ is (are):
\sc
\be
(0,0) \,\mbox{and}\,(10,2)
\ee
\be
(0,0)\,\mbox{only}
\ee
\be
(10,2)\,\mbox{only}
\ee
\be
(5,2)\,\mbox{only}
\ee
\be
(0,0) \,\mbox{and}\,(5,2)
\ee
\v2



\item %8
A particle moves according to the position function \\$s(t)=t^2\,e^{-t};\,t\geq 0,$ where $t$ is in seconds and $s$ is in meters. What is the total distance traveled by the particle during the first $6$ seconds?
\sc
\be
36\,e^{-6}
\ee
\be
4\,e^{-2}-36\,e^{-6}
\ee
\be
8\,e^{-2}
\ee
\be
4\,e^{-2}
\ee
\be
8\,e^{-2}-36\,e^{-6}
\ee

\newpage



\item %9
$\displaystyle \lim_{x\rightarrow\infty} \,x\,\sin\left(\frac{4}{x}\right)=$
\sc
\be
1
\ee
\be
\mbox{Does not exist}
\ee
\be
1/4
\ee
\be
0
\ee
\be
4
\ee
\v2



\item %10
If $\displaystyle h(x)=\frac{x\,\sin\,x}{g(x)};$ with $g\left(\displaystyle\frac{\pi}{2}\right)=1$ and $h'\left(\displaystyle\frac{\pi}{2}\right)=-2,$ then $g'\left(\displaystyle \frac{\pi}{2}\right)$ is :
\sc
\be
4/\pi
\ee
\be
-4/\pi
\ee
\be
0
\ee
\be
-6/\pi
\ee
\be
6/\pi
\ee

\newpage



\item %11
$\displaystyle \lim_{\theta\rightarrow 0} \frac{\sin(2\theta)-\cos(5\,\theta)+\tan(3\,\theta)+1}{\sin(3\,\theta)+\cos(6\,\theta)-\tan(2\theta)-1}=$
\sc
\be
3
\ee
\be
\mbox{Does not exist}
\ee
\be
2
\ee
\be
5
\ee
\be
4
\ee
\v2



\item %12
The function $y=f(x)$ satisfies the equation\\ $xy''+y'+xy=0,$ for all $x$. If $f(0)=1,$ then $f''(0)=$
\sc
\be
\displaystyle \frac{1}{2}
\ee
\be
2
\ee
\be
3
\ee
\be
\displaystyle \frac{-1}{2}
\ee
\be
\displaystyle \frac{-3}{4}
\ee

\newpage



\item %13
If $\displaystyle y=\tan^{-1} \left(\frac{b+a\cos\,x}{a-b\cos\,x}\right),$ where $a$ and $b$ are non-zero constants, then $\displaystyle \frac{dy}{dx}=$
\sc
\be
\displaystyle \frac{a^2-b^2}{(a-b\,\cos\,x)^2}
\ee
\be
\displaystyle \frac{ab\,\sin\,x}{(a-b\,\cos\,x)^2}
\ee
\be
\displaystyle \frac{-\sin\,x}{1+\cos^{2}\,x}
\ee
\be
\displaystyle 0
\ee
\be
\displaystyle \frac{a^2\,\cos\,x}{b^2 (1+\sin^{2}\,x)}
\ee
\v2



\item %14
If the curves $y=3-x^2$ and $y=Ax^3+B$ intersect at $(1,2)$ and their tangent lines at that point are perpendicular, then $7A+B=$
\sc
\be
1
\ee
\be
3
\ee
\be
2
\ee
\be
0
\ee
\be
4
\ee

\newpage



\item %15
If $F(x)=3f(4e^{x})\cdot\cos\,x,$ and $\displaystyle f'(4)=\frac{-1}{2}$ then $F'(0)=?$
\sc
\be
6
\ee
\be
4
\ee
\be
-3
\ee
\be
-6
\ee
\be
3
\ee
\v2



\item %16
The equation of the normal line to $y=\sin(\cos\,x)$ at $\displaystyle \left(\frac{\pi}{2},0\right)$
\sc
\be
y=\displaystyle \frac{\pi}{2}\,x+1
\ee
\be
y=-x\displaystyle +\frac{\pi}{2}
\ee
\be
y=x-\displaystyle \frac{\pi}{2}
\ee
\be
y=x+\displaystyle \frac{\pi}{2}
\ee
\be
y=-x\displaystyle -\frac{\pi}{2}
\ee

\newpage



\item %17
If $f(0)=1$ and $f'(0)=2,$ then $(f^{-1})'(1)=$
\sc
\be
\displaystyle \frac{1}{2}
\ee
\be
\displaystyle \frac{-1}{2}
\ee
\be
\displaystyle \frac{-1}{4}
\ee
\be
-1
\ee
\be
\displaystyle \frac{1}{4}
\ee
\v2



\item %18
A piece of land is shaped like a right triangle. Two people start at the right angle of the triangle at the same time, and walk at the same speed along the two different legs of the triangle. If the area of the triangle formed by the positions of the two people and their starting point (the right angle) is changing at $4\,m^2/s,$ then how fast are the two people moving when they are $5m$ from the right angle?
\sc
\be
2.0\,m/s
\ee
\be
2.4\,m/s
\ee
\be
0.8\,m/s
\ee
\be
1.6\,m/s
\ee
\be
0.4\,m/s
\ee

\newpage



\item %19
If $\displaystyle f(x)=xe^{-x},$ then $f^{(n)}(0)= $\\[0.in] $(f^{(n)}(x): n^{th}\,\mbox{derivative of $f$ at}\, x)$
\sc
\be
(-1)^{n}\cdot (n+1)
\ee
\be
(-1)^{n}\cdot n
\ee
\be
(-1)^{n+1}\cdot n
\ee
\be
(-1)^{n}\cdot (2n)
\ee
\be
(-1)^{n+1}\cdot (n+1)
\ee
\v2



\item %20
If $y=(2x)^{4x},$ then $y'=$
\sc
\be
4xy(1+\ln\,(4x))
\ee
\be
4y(1+\ln\,(2x))
\ee
\be
4y(x+\ln\,(2x))
\ee
\be
(4x)\cdot(2x)^{4x-1}
\ee
\be
8y\,\ln(2x)
\ee

\newpage



\en
\end{large}

\newpage


\renewcommand{\thepage}{\noindent Math 101, Term 171, Exam II \hfill Answer Sheet  \hfill {\bf \fbox{004}}}

\begin{Large}


\begin{tabular}{llll}
Name & .................................................& & \\
ID &   ................................& Sec & ..........\\
\end{tabular}

\vspace{10mm}


\end{Large}
\begin{normalsize}
\begin{center}
\begin{tabular}{|c|c c c c c c|c|c|c c c c c c|c|c|c c c c c c|}
\cline{1-7}\cline{9-15}
1  & a & b & c & d & e & f & \raisebox{0ex}[0cm][0cm]{\hspace{1cm}} & 36 & a & b & c & d & e & f\\ \cline{1-7}\cline{9-15}
2 & a & b & c & d & e & f & & 37& a & b & c & d & e & f\\ \cline{1-7}\cline{9-15}
3 & a & b & c & d & e & f & & 38& a & b & c & d & e & f\\ \cline{1-7}\cline{9-15}
4 & a & b & c & d & e & f & & 39& a & b & c & d & e & f\\ \cline{1-7}\cline{9-15}
5 & a & b & c & d & e & f & & 40& a & b & c & d & e & f\\ \cline{1-7}\cline{9-15}
6 & a & b & c & d & e & f & & 41& a & b & c & d & e & f\\ \cline{1-7}\cline{9-15}
7 & a & b & c & d & e & f & & 42& a & b & c & d & e & f\\ \cline{1-7}\cline{9-15}
8 & a & b & c & d & e & f & & 43& a & b & c & d & e & f\\ \cline{1-7}\cline{9-15}
9 & a & b & c & d & e & f & & 44& a & b & c & d & e & f\\ \cline{1-7}\cline{9-15}
10 & a & b & c & d & e & f & & 45& a & b & c & d & e & f\\ \cline{1-7}\cline{9-15}
11 & a & b & c & d & e & f & & 46& a & b & c & d & e & f\\ \cline{1-7}\cline{9-15}
12 & a & b & c & d & e & f & & 47& a & b & c & d & e & f\\ \cline{1-7}\cline{9-15}
13 & a & b & c & d & e & f & & 48& a & b & c & d & e & f\\ \cline{1-7}\cline{9-15}
14 & a & b & c & d & e & f & & 49& a & b & c & d & e & f\\ \cline{1-7}\cline{9-15}
15 & a & b & c & d & e & f & & 50& a & b & c & d & e & f\\ \cline{1-7}\cline{9-15}
16 & a & b & c & d & e & f & & 51& a & b & c & d & e & f\\ \cline{1-7}\cline{9-15}
17 & a & b & c & d & e & f & & 52& a & b & c & d & e & f\\ \cline{1-7}\cline{9-15}
18 & a & b & c & d & e & f & & 53& a & b & c & d & e & f\\ \cline{1-7}\cline{9-15}
19 & a & b & c & d & e & f & & 54& a & b & c & d & e & f\\ \cline{1-7}\cline{9-15}
20 & a & b & c & d & e & f & & 55& a & b & c & d & e & f\\ \cline{1-7}\cline{9-15}
21 & a & b & c & d & e & f & & 56& a & b & c & d & e & f\\ \cline{1-7}\cline{9-15}
22 & a & b & c & d & e & f & & 57& a & b & c & d & e & f\\ \cline{1-7}\cline{9-15}
23 & a & b & c & d & e & f & & 58& a & b & c & d & e & f\\ \cline{1-7}\cline{9-15}
24 & a & b & c & d & e & f & & 59& a & b & c & d & e & f\\ \cline{1-7}\cline{9-15}
25 & a & b & c & d & e & f & & 60& a & b & c & d & e & f\\ \cline{1-7}\cline{9-15}
26 & a & b & c & d & e & f & & 61& a & b & c & d & e & f\\ \cline{1-7}\cline{9-15}
27 & a & b & c & d & e & f & & 62& a & b & c & d & e & f\\ \cline{1-7}\cline{9-15}
28 & a & b & c & d & e & f & & 63& a & b & c & d & e & f\\ \cline{1-7}\cline{9-15}
29 & a & b & c & d & e & f & & 64& a & b & c & d & e & f\\ \cline{1-7}\cline{9-15}
30 & a & b & c & d & e & f & & 65& a & b & c & d & e & f\\ \cline{1-7}\cline{9-15}
31 & a & b & c & d & e & f & & 66& a & b & c & d & e & f\\ \cline{1-7}\cline{9-15}
32 & a & b & c & d & e & f & & 67& a & b & c & d & e & f\\ \cline{1-7}\cline{9-15}
33 & a & b & c & d & e & f & & 68& a & b & c & d & e & f\\ \cline{1-7}\cline{9-15}
34 & a & b & c & d & e & f & & 69& a & b & c & d & e & f\\ \cline{1-7}\cline{9-15}
35 & a & b & c & d & e & f & & 70& a & b & c & d & e & f\\ \cline{1-7}\cline{9-15}
\end{tabular}\end{center}
\end{normalsize}
\newpage
\renewcommand{\thepage}{\noindent Math 101, Term 171, Exam II \hfill \arabic{page} \hfill {\bf \fbox{ANSWER KEY}}}
\begin{normalsize}
\setcounter{page}{1}
\vspace {1cm}


\begin{center}
\begin{tabular}{|c||c|c|c|c|c|}
\hline
Q & MM & V1& V2& V3& V4\\
\hline \hline
1 & a  & d & c & c & e \\ \hline 
2 & a  & d & c & a & e \\ \hline 
3 & a  & e & b & d & a \\ \hline 
4 & a  & c & c & a & a \\ \hline 
5 & a  & b & e & c & a \\ \hline 
6 & a  & c & d & e & d \\ \hline 
7 & a  & a & a & c & a \\ \hline 
8 & a  & b & a & d & e \\ \hline 
9 & a  & d & c & c & e \\ \hline 
10 & a  & e & e & e & e \\ \hline 
11 & a  & d & a & b & d \\ \hline 
12 & a  & d & d & c & d \\ \hline 
13 & a  & a & b & b & c \\ \hline 
14 & a  & a & e & e & b \\ \hline 
15 & a  & c & a & c & d \\ \hline 
16 & a  & c & e & c & c \\ \hline 
17 & a  & b & d & b & a \\ \hline 
18 & a  & d & a & a & c \\ \hline 
19 & a  & d & a & d & c \\ \hline 
20 & a  & b & c & a & b \\ \hline 
\end{tabular}
\end{center}
\newpage
\end{normalsize}
\renewcommand{\thepage}{\noindent Math 101, Term 171, Exam II \hfill \arabic{page} \hfill {\bf \fbox{Answer Counts}}}

\begin{normalsize}
\begin{center}
\vspace {1cm}

\begin{Large}
Answer Counts \\
\end{Large}
\vspace {1cm}
\begin{tabular}{|c||c|c|c|c|c|c|}
\hline
V & a & b & c & d & e \\ \hline \hline
1 & 4 & 2 & 3 & 9 & 2 \\ \hline
2 & 2 & 4 & 7 & 3 & 4 \\ \hline
3 & 3 & 3 & 4 & 7 & 3 \\ \hline
4 & 2 & 6 & 3 & 2 & 7 \\ \hline
\end{tabular}
\end{center}
\end{normalsize}
\newpage
\end{document}
