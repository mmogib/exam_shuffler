\documentclass[amsfonts,bezier,leqno,fleqn,12pt,a4paper]{article}

\begin{document}

\topmargin=-.75in
\textheight=26.5cm
\footskip=.3in

\oddsidemargin=-.1cm
\textwidth=16.95cm

\arraycolsep=.15in
\labelsep=.75cm

\pagestyle{myheadings}

\def \al{\alpha}
\def \b{\beta}
\def \ba{\begin{array}}
\def \del{\delta}
\def \Del{\Delta}
\def \ds{\displaystyle}
\def \fr{\frac}
\def \hf{\hfill}
\def \hl{\hline}
\def \hrf{\hrulefill}
\def \hs1{\hspace*{4mm}}
\def \inf{\infty}
\def \lam{\lambda}
\def \lan{\langle}
\def \lb{\linebreak}
\def \l{\left}
\def \la{\leftarrow}
\def \mb{\mbox}
\def \no1{\noindent}
\def \n1{\newpage}
\def \ov{\overline}
\def \p{\put}
\def \ran{\rangle}
\def \ra{\rightarrow}
\def \r{\right}
\def \s1{\sqrt}
\def \ti{\times}
\def \tr{\triangle}
\def \ts{\textstyle}
\def \th{\theta}
\def \ul{\unitlength}
\def \un1{\underline}
\def \vs1{\vspace {4mm}}
\def \v2{\vspace {3.5cm}}
\def \z{\overline{z}}

\def \bi{\begin{itemize}}
\def \ei{\end{itemize}}
\def \bt{\begin{tabular}}
\def  \et{\end{tabular}}
\def \bp{\begin{picture}}
\def \ep{\end{picture}}
\def \bc{\begin{center}}
\def \ec{\end{center}}
\def \ea{\end{array}}
\def \ba{\begin{array}}
\def \be{\begin{equation}}
\def \ee{\end{equation}}
\def \bn{\begin{enumerate}}
\def \en{\end{enumerate}}

\renewcommand{\sc}{\vspace {0.3in} \setcounter{equation}{0}}
\renewcommand{\theequation}{\alph{equation}}
\newtheorem{question}{\mbox{}}


\thicklines
\pagestyle{myheadings}

\thispagestyle{empty}

\begin{center}
\begin{large}

King Fahd University of Petroleum and Minerals\\
Department of Mathematics and Statistics\\
\vspace*{2cm}
{\bf Math 101}  \\
{\bf Final Exam}  \\
{\bf Term 162}  \\
{\bf Friday 26/5/2017}  \\

\vspace*{3cm}
{\bf{\Huge{\fbox{EXAM COVER}}}}\\
\vspace*{2cm}
{\bf Number of versions: 4 }  \\
{\bf Number of questions: 28 }  \\
{\bf Number of Answers: 5 per question}  \\

\end{large}
\vfill

\tiny{This exam was prepared using mcqs\\}
\tiny{For questions send an email to Dr. Ibrahim Al-Lehyani (iallehyani@kaau.edu.sa)\\}

\end{center}

\newpage


\thispagestyle{empty}

\begin{center}
\begin{large}

King Fahd University of Petroleum and Minerals\\
Department of Mathematics and Statistics\\
\vspace*{4cm}
{\bf Math 101} \\
{\bf Final Exam}  \\
{\bf Term 162}  \\
{\bf Friday 26/5/2017}  \\
{\bf Net Time Allowed: 180 minutes}  \\
\vspace*{6cm}
{\bf {\Huge{MASTER VERSION}}}  \\

\end{large}
\end{center}

\newpage


\renewcommand{\thepage}{\noindent Math 101, Final Exam, Term 162 \hfill Page {\bf \arabic{page} of 14} \hfill {\bf \fbox{MASTER}}}

\setcounter{page}{1}

\begin{large}
\bn


\item %1
Let $\displaystyle h(x)=\frac{g(x)}{f(x)+g(x)}$. If $f(4)=1$, $g(4)=2$, $f'(4)=3$, and $g'(4)=-3$, then $h'(4)=$
\sc
\be
-1
\ee
\be
0
\ee
\be
-3
\ee
\be
3
\ee
\be
-2
\ee
\v2



\item %2
A particle moves in a straight line and has acceleration given by $a(t)=4t+10.$ Its initial velocity is $v(0)=-3$ feet/sec and its initial displacement $s(0)=5$ feet, its position is given by
\sc
\be
\displaystyle s(t)=\frac{2}{3}\,t^3+5t^2-3t+5
\ee
\be
\displaystyle s(t)=\frac{2}{3}\,t^5+5t^2+3t-5
\ee
\be
\displaystyle s(t)=3t^3+10t^2-3t+5
\ee
\be
\displaystyle s(t)=3t^3+10t^2+3t-5
\ee
\be
\displaystyle s(t)=\frac{2}{3}\,t^3+5t^2-3t
\ee
\newpage



\item %3
Let $f(x)=c\,x+\ln(\cos x)$ where $c$ is a constant. The value of $c$ such that $f'(\displaystyle \frac{\pi}{4})=6$ equals to
\sc
\be
7
\ee
\be
6
\ee
\be
-2
\ee
\be
1
\ee
\be
0
\ee
\v2



\item %4
$\displaystyle \tanh \, (\ln\,x)=$
\sc
\be
\displaystyle \frac{x^2-1}{x^2+1}
\ee
\be
\displaystyle \frac{x^2+1}{x^2-1}
\ee
\be
\displaystyle \frac{\ln\,x^2-1}{\ln\,x^2+1}
\ee
\be
\displaystyle \frac{x-1}{x+1}
\ee
\be
0
\ee
\newpage



\item %5
Suppose $f''$ is continuous on $(-\infty,\infty).$ If $f'(2)=0$ and $f''(2)=-5$ then
\sc
\be
f\,\mbox{has a local maximum at}\,x=2
\ee
\be
f\,\mbox{has a local minimum at}\,x=2
\ee
\be
f\,\mbox{has a point of inflection at}\,x=2
\ee
\be
f\,\mbox{is increasing at}\,x=2
\ee
\be
f\,\mbox{is concave upward at}\,x=2
\ee
\v2



\item %6
The sum of all positive real number $a$, that makes the function\\[0.2in] $\displaystyle f(x)= \left\{\begin{array}{lll} ax+3 & \mbox{if}\,x>a\\ \displaystyle x^2-x+2a^2 & \mbox{if}\,x\leq a  \\ \end{array}\right.$ continuous everywhere, is
\sc
\be
\displaystyle \frac{3}{2}
\ee
\be
\displaystyle 1
\ee
\be
\displaystyle \frac{5}{2}
\ee
\be
2
\ee
\be
3
\ee
\newpage



\item %7
If $\displaystyle \cosh\,x=\frac{5}{3}$ and $x>0,$ then
\sc
\be
\displaystyle \tanh\,x=\frac{4}{5}
\ee
\be
\displaystyle \coth\,x=\frac{3}{5}
\ee
\be
\displaystyle \sinh\,x=\frac{3}{4}
\ee
\be
\displaystyle \mbox{csch}\,x=\frac{4}{3}
\ee
\be
\displaystyle \mbox{sech}\, x=1
\ee
\v2



\item %8
Let $\displaystyle f(x)=x\sqrt{9-x^2}$ on the interval $[-3,3]$. The function $f$ attains its absolute maximum value at
\sc
\be
\displaystyle x=\displaystyle\frac{3}{\sqrt{2}}
\ee
\be
\displaystyle x=3
\ee
\be
\displaystyle x=\displaystyle-\frac{3}{\sqrt{2}}
\ee
\be
x=-3
\ee
\be
x=0
\ee
\newpage



\item %9
The critical number(s) for the function $\displaystyle f(x)=x^2\,\ln\,x$ is(are)
\sc
\be
\displaystyle x=\displaystyle\frac{1}{\sqrt{e}}
\ee
\be
x=1
\ee
\be
\displaystyle x=\sqrt{e}
\ee
\be
\displaystyle x=\frac{1}{e}
\ee
\be
\displaystyle x=1\,\mbox{and}\,\sqrt{e}
\ee
\v2



\item %10
$\displaystyle \lim_{x\rightarrow 0}\, \left[ \frac{\sqrt{1+2x}-\sqrt{1-4x}}{x}\right]=$
\sc
\be
3
\ee
\be
6
\ee
\be
2
\ee
\be
1
\ee
\be
0
\ee
\newpage



\item %11
$\displaystyle \lim_{x\rightarrow 1^+}\, \left[ \ln(x^8-1)-\ln(x^4-1)\right]=$
\sc
\be
\ln\,2
\ee
\be
\displaystyle \ln\, \left(\displaystyle\frac{1}{2}\right)
\ee
\be
\ln\,32
\ee
\be
\displaystyle \ln\, \left(\displaystyle\frac{1}{32}\right)
\ee
\be
0
\ee
\v2



\item %12
If $\displaystyle y=\ln\,(1+\ln\,x)$ and $x>\,e$, then $y''=$
\sc
\be
\displaystyle \frac{-2-\ln\,x}{x^2(1+\ln\,x)^2}
\ee
\be
\displaystyle \frac{-1}{x(1+\ln\,x)}
\ee
\be
\displaystyle \frac{1}{x(1+\ln\,x)}
\ee
\be
\displaystyle \frac{2+\ln\,x}{x^2(1+\ln\,x)^2}
\ee
\be
\displaystyle \frac{1}{1+\ln\,x}
\ee
\newpage



\item %13
The area of the largest rectangle that can be inscribed in a circle of radius $1$ is
\sc
\be
2
\ee
\be
\displaystyle \frac{2}{\sqrt{2}}
\ee
\be
4
\ee
\be
\displaystyle 2\,\sqrt{2}
\ee
\be
2\,\pi
\ee
\v2



\item %14
If $y\,\sec x=x\, \tan y \,+\, 1$, then $\displaystyle {\frac{dy}{dx}}$ when $x=0$ equals
\sc
\be
\displaystyle \tan\,1
\ee
\be
\displaystyle \frac{\sec\,1}{\tan\,1}
\ee
\be
\displaystyle \sec\, 1
\ee
\be
\displaystyle \frac{1+\tan\,1}{\sec\,1}
\ee
\be
1
\ee
\newpage



\item %15
The linearization  $L(x)$ of the function $f(x)=\sin\,x$ at \\$a=\pi/6$ is
\sc
\be
\displaystyle L(x)=\frac{1}{2}+\frac{\sqrt{3}}{2} \left(x-\frac{\pi}{6}\right)
\ee
\be
\displaystyle L(x)=\frac{\sqrt{3}}{2}+\frac{1}{2} \left(x-\frac{\pi}{6}\right)
\ee
\be
\displaystyle L(x)=1+\left(x-\frac{\pi}{6}\right)
\ee
\be
\displaystyle L(x)=1+x
\ee
\be
\displaystyle L(x)=x-\frac{\pi}{6}
\ee
\v2



\item %16
If $y=\sqrt{\sin \,x+\,y},$ then $\displaystyle \frac{dy}{dx}=$
\sc
\be
\displaystyle \frac{\cos\,x}{2y-1}
\ee
\be
\displaystyle \frac{\cos\,x}{2y}
\ee
\be
\displaystyle \frac{\sin\, x}{2y-1}
\ee
\be
\displaystyle \frac{\cos\,x}{1-y}
\ee
\be
\displaystyle \frac{\cos\,x}{1-2y}
\ee
\newpage



\item %17
Which one of the following statements is \textbf{TRUE}?
\sc
\be
.
\ee
\be
.
\ee
\be
.
\ee
\be
.
\ee
\be
.
\ee
\v2



\item %18
The slope of the tangent line to the graph of $y=(x^2+1)^3\,e^{x^2}$ at $x=1$ equals
\sc
\be
40\,e
\ee
\be
5
\ee
\be
20\,e
\ee
\be
8\,e
\ee
\be
5\,e
\ee
\newpage



\item %19
If $f(4)=-1$ and $f'(x)\geq 5$ for $2\leq x \leq 4,$ then the largest possible value of $f(2)$ is
\sc
\be
-11
\ee
\be
-1
\ee
\be
5
\ee
\be
\displaystyle -\frac{1}{5}
\ee
\be
-5
\ee
\v2



\item %20
The graph of the derivative $f'$ of a \textbf{continuous} function $f$ is shown below, then which one of the following statements is \textbf{TRUE}?\\[0.5in]
\sc
\be
f\,\mbox{has a local maximum at}\,x=2 \,\mbox{and}\, x=6
\ee
\be
f\,\mbox{is increasing on the intervals}\,(0,2),(4,6),\,\mbox{and}\,(6,8)
\ee
\be
f\, \mbox{has a local minimum at}\,x=4 \,\mbox{and}\, x=6
\ee
\be
f\, \mbox{is decreasing on the intervals}\, (0,2), (4,6),\,\mbox{and}\,(6,8)
\ee
\be
f\, \mbox{has an inflection point at}\,x=6
\ee
\newpage



\item %21
Using a linear approximation (or differentials), the best estimation to $\sqrt[3]{1001}$ is $10+B.$ Then $B=$
\sc
\be
\displaystyle \frac{1}{300}
\ee
\be
\displaystyle \frac{3}{10}
\ee
\be
\displaystyle \frac{1}{3}
\ee
\be
\displaystyle \frac{5}{300}
\ee
\be
\displaystyle \frac{1}{100}
\ee
\v2



\item %22
Let $A$ and $B$ be constants such that $y=A\,\sin\,x+B\,\cos\,x$ satisfies the equation $y''+y'-2y=\sin\,x$. Then
\sc
\be
\displaystyle A=\frac{-3}{10}\,\mbox{and}\,B=\frac{-1}{10}
\ee
\be
\displaystyle A=\frac{2}{5}\,\mbox{and}\,B=\frac{3}{5}
\ee
\be
\displaystyle A=\frac{2}{3}\,\mbox{and}\,B=\frac{5}{2}
\ee
\be
\displaystyle A=\frac{-1}{3}\,\mbox{and}\,B=-1
\ee
\be
A=1\,\mbox{and}\, B=-1
\ee
\newpage



\item %23
Which one of the following graphs represents the function $\displaystyle f(x)=\frac{x^{2}+1}{x}?$
\sc
\be
\mbox{graph}
\ee
\be
\mbox{graph}
\ee
\be
\mbox{graph}
\ee
\be
\mbox{graph}
\ee
\be
\mbox{graph}
\ee
\v2



\item %24
Which one of the following statements is \textbf{FALSE}?
\sc
\be
.
\ee
\be
.
\ee
\be
.
\ee
\be
.
\ee
\be
.
\ee
\newpage



\item %25
Let $f(x)=x^3-6x^2+9x+1$ and $x\in[2,4].$ If $M$ is the absolute maximum value and $m$ is the absolute minimum value, then $M+m=$
\sc
\be
6
\ee
\be
8
\ee
\be
4
\ee
\be
2
\ee
\be
1
\ee
\v2



\item %26
One inflection point of the graph of $y=e^x\,\sin x$ on $[-\pi,\,\pi]$ is
\sc
\be
\displaystyle \left(\frac{\pi}{2},\, e^{\frac{\pi}{2}}\right)
\ee
\be
\displaystyle \left(\frac{\pi}{6},\, \frac{e^{\frac{\pi}{6}}}{2}\right)
\ee
\be
\displaystyle \left(\frac{\pi}{3},\, \frac{\sqrt{3}\, e^{\frac{\pi}{2}}}{2}\right)
\ee
\be
(\pi,0)
\ee
\be
(0,0)
\ee
\newpage



\item %27
The graph of $y=(1-x)\,e^x$ is concave downward on
\sc
\be
\displaystyle (-1,\infty)
\ee
\be
(-1,0)
\ee
\be
(-\infty,1)
\ee
\be
(0,1)
\ee
\be
(-e,-1)
\ee
\v2



\item %28
Suppose $f(x)= \left\{\begin{array}{lll} 2x & \mbox{for} & x\leq 0 \\ \displaystyle 2x-2x^2 & \mbox{for} & 0 < x <2\\ 2-6x & \mbox{for} & x\geq 2 \end{array}\right.$.\\ The function $f$ is \textbf{not differentiable} when
\sc
\be
x=2
\ee
\be
x=0
\ee
\be
x\in(0,2)
\ee
\be
x\in(-\infty,0]
\ee
\be
x=0\,\mbox{and}\,x=2
\ee
\newpage



\en

\end{large}

\newpage


\thispagestyle{empty}

\begin{center}
\begin{large}

King Fahd University of Petroleum and Minerals\\
Department of Mathematics and Statistics\\
\vspace*{0.5cm}
{\bf \fbox{CODE {\small{001}}}} \hfill {\bf Math 101} \hfill {\bf \fbox{CODE {\small{001}}}} \\
{\bf Final Exam}  \\
{\bf Term 162}  \\
{\bf Friday 26/5/2017}  \\
{\bf Net Time Allowed: 180 minutes}  \\
\vspace*{0.2cm}

\end{large}
\end{center}

\large{Name:  }\hrulefill

\vspace{3mm}

\large{ID: } \hrulefill \large{  Sec: } \hrulefill \large{.

\vspace{1cm}

\large{\bf{Check that this exam has {\underline{28}} questions.}}

\vspace{1cm}

\underline{{\large{\bf Important Instructions:}}}

\begin{enumerate}
\begin{normalsize}
\item  All types of calculators, pagers or mobile phones are NOT allowed during the examination.
\item  Use HB 2.5 pencils only.
\item  Use a good eraser. DO NOT use the erasers attached to the pencil.
\item  Write your name, ID number and Section number on the examination paper and in the upper left corner of the answer sheet.
\item  When bubbling your ID number and Section number, be sure that the bubbles match with the numbers that you write.
\item  The Test Code Number is already bubbled in your answer sheet. Make sure that it is the same as that printed on your question paper.
\item  When bubbling, make sure that the bubbled space is fully covered.
\item  When erasing a bubble, make sure that you do not leave any trace of penciling.
\end{normalsize}
\end{enumerate}

\newpage


\renewcommand{\thepage}{\noindent
Math 101, Final Exam, Term 162 \hfill Page {\bf \arabic{page} of 14} \hfill {\bf \fbox{001}}}

\setcounter{page}{1}

\begin{large}

\bn


\item %1
Suppose $f''$ is continuous on $(-\infty,\infty).$ If $f'(2)=0$ and $f''(2)=-5$ then
\sc
\be
f\,\mbox{has a local maximum at}\,x=2
\ee
\be
f\,\mbox{has a point of inflection at}\,x=2
\ee
\be
f\,\mbox{has a local minimum at}\,x=2
\ee
\be
f\,\mbox{is increasing at}\,x=2
\ee
\be
f\,\mbox{is concave upward at}\,x=2
\ee
\v2



\item %2
The critical number(s) for the function $\displaystyle f(x)=x^2\,\ln\,x$ is(are)
\sc
\be
x=1
\ee
\be
\displaystyle x=1\,\mbox{and}\,\sqrt{e}
\ee
\be
\displaystyle x=\sqrt{e}
\ee
\be
\displaystyle x=\displaystyle\frac{1}{\sqrt{e}}
\ee
\be
\displaystyle x=\frac{1}{e}
\ee

\newpage



\item %3
$\displaystyle \tanh \, (\ln\,x)=$
\sc
\be
\displaystyle \frac{x^2-1}{x^2+1}
\ee
\be
\displaystyle \frac{x^2+1}{x^2-1}
\ee
\be
\displaystyle \frac{\ln\,x^2-1}{\ln\,x^2+1}
\ee
\be
0
\ee
\be
\displaystyle \frac{x-1}{x+1}
\ee
\v2



\item %4
Let $\displaystyle h(x)=\frac{g(x)}{f(x)+g(x)}$. If $f(4)=1$, $g(4)=2$, $f'(4)=3$, and $g'(4)=-3$, then $h'(4)=$
\sc
\be
3
\ee
\be
-3
\ee
\be
0
\ee
\be
-1
\ee
\be
-2
\ee

\newpage



\item %5
The sum of all positive real number $a$, that makes the function\\[0.2in] $\displaystyle f(x)= \left\{\begin{array}{lll} ax+3 & \mbox{if}\,x>a\\ \displaystyle x^2-x+2a^2 & \mbox{if}\,x\leq a  \\ \end{array}\right.$ continuous everywhere, is
\sc
\be
2
\ee
\be
3
\ee
\be
\displaystyle \frac{3}{2}
\ee
\be
\displaystyle 1
\ee
\be
\displaystyle \frac{5}{2}
\ee
\v2



\item %6
Let $\displaystyle f(x)=x\sqrt{9-x^2}$ on the interval $[-3,3]$. The function $f$ attains its absolute maximum value at
\sc
\be
x=0
\ee
\be
x=-3
\ee
\be
\displaystyle x=\displaystyle-\frac{3}{\sqrt{2}}
\ee
\be
\displaystyle x=\displaystyle\frac{3}{\sqrt{2}}
\ee
\be
\displaystyle x=3
\ee

\newpage



\item %7
A particle moves in a straight line and has acceleration given by $a(t)=4t+10.$ Its initial velocity is $v(0)=-3$ feet/sec and its initial displacement $s(0)=5$ feet, its position is given by
\sc
\be
\displaystyle s(t)=\frac{2}{3}\,t^3+5t^2-3t+5
\ee
\be
\displaystyle s(t)=3t^3+10t^2-3t+5
\ee
\be
\displaystyle s(t)=\frac{2}{3}\,t^5+5t^2+3t-5
\ee
\be
\displaystyle s(t)=3t^3+10t^2+3t-5
\ee
\be
\displaystyle s(t)=\frac{2}{3}\,t^3+5t^2-3t
\ee
\v2



\item %8
Let $f(x)=c\,x+\ln(\cos x)$ where $c$ is a constant. The value of $c$ such that $f'(\displaystyle \frac{\pi}{4})=6$ equals to
\sc
\be
1
\ee
\be
-2
\ee
\be
6
\ee
\be
0
\ee
\be
7
\ee

\newpage



\item %9
If $\displaystyle \cosh\,x=\frac{5}{3}$ and $x>0,$ then
\sc
\be
\displaystyle \coth\,x=\frac{3}{5}
\ee
\be
\displaystyle \mbox{sech}\, x=1
\ee
\be
\displaystyle \mbox{csch}\,x=\frac{4}{3}
\ee
\be
\displaystyle \tanh\,x=\frac{4}{5}
\ee
\be
\displaystyle \sinh\,x=\frac{3}{4}
\ee
\v2



\item %10
$\displaystyle \lim_{x\rightarrow 0}\, \left[ \frac{\sqrt{1+2x}-\sqrt{1-4x}}{x}\right]=$
\sc
\be
3
\ee
\be
6
\ee
\be
1
\ee
\be
2
\ee
\be
0
\ee

\newpage



\item %11
The slope of the tangent line to the graph of $y=(x^2+1)^3\,e^{x^2}$ at $x=1$ equals
\sc
\be
8\,e
\ee
\be
40\,e
\ee
\be
5
\ee
\be
20\,e
\ee
\be
5\,e
\ee
\v2



\item %12
If $\displaystyle y=\ln\,(1+\ln\,x)$ and $x>\,e$, then $y''=$
\sc
\be
\displaystyle \frac{1}{1+\ln\,x}
\ee
\be
\displaystyle \frac{-2-\ln\,x}{x^2(1+\ln\,x)^2}
\ee
\be
\displaystyle \frac{2+\ln\,x}{x^2(1+\ln\,x)^2}
\ee
\be
\displaystyle \frac{-1}{x(1+\ln\,x)}
\ee
\be
\displaystyle \frac{1}{x(1+\ln\,x)}
\ee

\newpage



\item %13
If $y\,\sec x=x\, \tan y \,+\, 1$, then $\displaystyle {\frac{dy}{dx}}$ when $x=0$ equals
\sc
\be
\displaystyle \frac{\sec\,1}{\tan\,1}
\ee
\be
\displaystyle \tan\,1
\ee
\be
1
\ee
\be
\displaystyle \sec\, 1
\ee
\be
\displaystyle \frac{1+\tan\,1}{\sec\,1}
\ee
\v2



\item %14
The area of the largest rectangle that can be inscribed in a circle of radius $1$ is
\sc
\be
2\,\pi
\ee
\be
4
\ee
\be
\displaystyle 2\,\sqrt{2}
\ee
\be
2
\ee
\be
\displaystyle \frac{2}{\sqrt{2}}
\ee

\newpage



\item %15
The graph of the derivative $f'$ of a \textbf{continuous} function $f$ is shown below, then which one of the following statements is \textbf{TRUE}?\\[0.5in]
\sc
\be
f\,\mbox{has a local maximum at}\,x=2 \,\mbox{and}\, x=6
\ee
\be
f\, \mbox{is decreasing on the intervals}\, (0,2), (4,6),\,\mbox{and}\,(6,8)
\ee
\be
f\,\mbox{is increasing on the intervals}\,(0,2),(4,6),\,\mbox{and}\,(6,8)
\ee
\be
f\, \mbox{has an inflection point at}\,x=6
\ee
\be
f\, \mbox{has a local minimum at}\,x=4 \,\mbox{and}\, x=6
\ee
\v2



\item %16
The linearization  $L(x)$ of the function $f(x)=\sin\,x$ at \\$a=\pi/6$ is
\sc
\be
\displaystyle L(x)=\frac{\sqrt{3}}{2}+\frac{1}{2} \left(x-\frac{\pi}{6}\right)
\ee
\be
\displaystyle L(x)=1+\left(x-\frac{\pi}{6}\right)
\ee
\be
\displaystyle L(x)=\frac{1}{2}+\frac{\sqrt{3}}{2} \left(x-\frac{\pi}{6}\right)
\ee
\be
\displaystyle L(x)=x-\frac{\pi}{6}
\ee
\be
\displaystyle L(x)=1+x
\ee

\newpage



\item %17
If $f(4)=-1$ and $f'(x)\geq 5$ for $2\leq x \leq 4,$ then the largest possible value of $f(2)$ is
\sc
\be
\displaystyle -\frac{1}{5}
\ee
\be
-5
\ee
\be
5
\ee
\be
-1
\ee
\be
-11
\ee
\v2



\item %18
Which one of the following statements is \textbf{TRUE}?
\sc
\be
.
\ee
\be
.
\ee
\be
.
\ee
\be
.
\ee
\be
.
\ee

\newpage



\item %19
If $y=\sqrt{\sin \,x+\,y},$ then $\displaystyle \frac{dy}{dx}=$
\sc
\be
\displaystyle \frac{\cos\,x}{2y-1}
\ee
\be
\displaystyle \frac{\cos\,x}{1-y}
\ee
\be
\displaystyle \frac{\cos\,x}{1-2y}
\ee
\be
\displaystyle \frac{\sin\, x}{2y-1}
\ee
\be
\displaystyle \frac{\cos\,x}{2y}
\ee
\v2



\item %20
$\displaystyle \lim_{x\rightarrow 1^+}\, \left[ \ln(x^8-1)-\ln(x^4-1)\right]=$
\sc
\be
\displaystyle \ln\, \left(\displaystyle\frac{1}{32}\right)
\ee
\be
\displaystyle \ln\, \left(\displaystyle\frac{1}{2}\right)
\ee
\be
\ln\,32
\ee
\be
\ln\,2
\ee
\be
0
\ee

\newpage



\item %21
Let $A$ and $B$ be constants such that $y=A\,\sin\,x+B\,\cos\,x$ satisfies the equation $y''+y'-2y=\sin\,x$. Then
\sc
\be
\displaystyle A=\frac{2}{3}\,\mbox{and}\,B=\frac{5}{2}
\ee
\be
A=1\,\mbox{and}\, B=-1
\ee
\be
\displaystyle A=\frac{-3}{10}\,\mbox{and}\,B=\frac{-1}{10}
\ee
\be
\displaystyle A=\frac{-1}{3}\,\mbox{and}\,B=-1
\ee
\be
\displaystyle A=\frac{2}{5}\,\mbox{and}\,B=\frac{3}{5}
\ee
\v2



\item %22
Which one of the following statements is \textbf{FALSE}?
\sc
\be
.
\ee
\be
.
\ee
\be
.
\ee
\be
.
\ee
\be
.
\ee

\newpage



\item %23
Using a linear approximation (or differentials), the best estimation to $\sqrt[3]{1001}$ is $10+B.$ Then $B=$
\sc
\be
\displaystyle \frac{1}{100}
\ee
\be
\displaystyle \frac{1}{300}
\ee
\be
\displaystyle \frac{3}{10}
\ee
\be
\displaystyle \frac{1}{3}
\ee
\be
\displaystyle \frac{5}{300}
\ee
\v2



\item %24
Which one of the following graphs represents the function $\displaystyle f(x)=\frac{x^{2}+1}{x}?$
\sc
\be
\mbox{graph}
\ee
\be
\mbox{graph}
\ee
\be
\mbox{graph}
\ee
\be
\mbox{graph}
\ee
\be
\mbox{graph}
\ee

\newpage



\item %25
One inflection point of the graph of $y=e^x\,\sin x$ on $[-\pi,\,\pi]$ is
\sc
\be
(0,0)
\ee
\be
\displaystyle \left(\frac{\pi}{6},\, \frac{e^{\frac{\pi}{6}}}{2}\right)
\ee
\be
(\pi,0)
\ee
\be
\displaystyle \left(\frac{\pi}{2},\, e^{\frac{\pi}{2}}\right)
\ee
\be
\displaystyle \left(\frac{\pi}{3},\, \frac{\sqrt{3}\, e^{\frac{\pi}{2}}}{2}\right)
\ee
\v2



\item %26
The graph of $y=(1-x)\,e^x$ is concave downward on
\sc
\be
\displaystyle (-1,\infty)
\ee
\be
(-1,0)
\ee
\be
(-e,-1)
\ee
\be
(-\infty,1)
\ee
\be
(0,1)
\ee

\newpage



\item %27
Let $f(x)=x^3-6x^2+9x+1$ and $x\in[2,4].$ If $M$ is the absolute maximum value and $m$ is the absolute minimum value, then $M+m=$
\sc
\be
2
\ee
\be
6
\ee
\be
8
\ee
\be
1
\ee
\be
4
\ee
\v2



\item %28
Suppose $f(x)= \left\{\begin{array}{lll} 2x & \mbox{for} & x\leq 0 \\ \displaystyle 2x-2x^2 & \mbox{for} & 0 < x <2\\ 2-6x & \mbox{for} & x\geq 2 \end{array}\right.$.\\ The function $f$ is \textbf{not differentiable} when
\sc
\be
x=0
\ee
\be
x=2
\ee
\be
x\in(-\infty,0]
\ee
\be
x=0\,\mbox{and}\,x=2
\ee
\be
x\in(0,2)
\ee

\newpage



\en
\end{large}

\newpage


\renewcommand{\thepage}{\noindent Math 101, Final Exam, Term 162 \hfill Answer Sheet  \hfill {\bf \fbox{001}}}

\begin{Large}


\begin{tabular}{llll}
Name & .................................................& & \\
ID &   ................................& Sec & ..........\\
\end{tabular}

\vspace{10mm}


\end{Large}
\begin{normalsize}
\begin{center}
\begin{tabular}{|c|c c c c c c|c|c|c c c c c c|c|c|c c c c c c|}
\cline{1-7}\cline{9-15}
1  & a & b & c & d & e & f & \raisebox{0ex}[0cm][0cm]{\hspace{1cm}} & 36 & a & b & c & d & e & f\\ \cline{1-7}\cline{9-15}
2 & a & b & c & d & e & f & & 37& a & b & c & d & e & f\\ \cline{1-7}\cline{9-15}
3 & a & b & c & d & e & f & & 38& a & b & c & d & e & f\\ \cline{1-7}\cline{9-15}
4 & a & b & c & d & e & f & & 39& a & b & c & d & e & f\\ \cline{1-7}\cline{9-15}
5 & a & b & c & d & e & f & & 40& a & b & c & d & e & f\\ \cline{1-7}\cline{9-15}
6 & a & b & c & d & e & f & & 41& a & b & c & d & e & f\\ \cline{1-7}\cline{9-15}
7 & a & b & c & d & e & f & & 42& a & b & c & d & e & f\\ \cline{1-7}\cline{9-15}
8 & a & b & c & d & e & f & & 43& a & b & c & d & e & f\\ \cline{1-7}\cline{9-15}
9 & a & b & c & d & e & f & & 44& a & b & c & d & e & f\\ \cline{1-7}\cline{9-15}
10 & a & b & c & d & e & f & & 45& a & b & c & d & e & f\\ \cline{1-7}\cline{9-15}
11 & a & b & c & d & e & f & & 46& a & b & c & d & e & f\\ \cline{1-7}\cline{9-15}
12 & a & b & c & d & e & f & & 47& a & b & c & d & e & f\\ \cline{1-7}\cline{9-15}
13 & a & b & c & d & e & f & & 48& a & b & c & d & e & f\\ \cline{1-7}\cline{9-15}
14 & a & b & c & d & e & f & & 49& a & b & c & d & e & f\\ \cline{1-7}\cline{9-15}
15 & a & b & c & d & e & f & & 50& a & b & c & d & e & f\\ \cline{1-7}\cline{9-15}
16 & a & b & c & d & e & f & & 51& a & b & c & d & e & f\\ \cline{1-7}\cline{9-15}
17 & a & b & c & d & e & f & & 52& a & b & c & d & e & f\\ \cline{1-7}\cline{9-15}
18 & a & b & c & d & e & f & & 53& a & b & c & d & e & f\\ \cline{1-7}\cline{9-15}
19 & a & b & c & d & e & f & & 54& a & b & c & d & e & f\\ \cline{1-7}\cline{9-15}
20 & a & b & c & d & e & f & & 55& a & b & c & d & e & f\\ \cline{1-7}\cline{9-15}
21 & a & b & c & d & e & f & & 56& a & b & c & d & e & f\\ \cline{1-7}\cline{9-15}
22 & a & b & c & d & e & f & & 57& a & b & c & d & e & f\\ \cline{1-7}\cline{9-15}
23 & a & b & c & d & e & f & & 58& a & b & c & d & e & f\\ \cline{1-7}\cline{9-15}
24 & a & b & c & d & e & f & & 59& a & b & c & d & e & f\\ \cline{1-7}\cline{9-15}
25 & a & b & c & d & e & f & & 60& a & b & c & d & e & f\\ \cline{1-7}\cline{9-15}
26 & a & b & c & d & e & f & & 61& a & b & c & d & e & f\\ \cline{1-7}\cline{9-15}
27 & a & b & c & d & e & f & & 62& a & b & c & d & e & f\\ \cline{1-7}\cline{9-15}
28 & a & b & c & d & e & f & & 63& a & b & c & d & e & f\\ \cline{1-7}\cline{9-15}
29 & a & b & c & d & e & f & & 64& a & b & c & d & e & f\\ \cline{1-7}\cline{9-15}
30 & a & b & c & d & e & f & & 65& a & b & c & d & e & f\\ \cline{1-7}\cline{9-15}
31 & a & b & c & d & e & f & & 66& a & b & c & d & e & f\\ \cline{1-7}\cline{9-15}
32 & a & b & c & d & e & f & & 67& a & b & c & d & e & f\\ \cline{1-7}\cline{9-15}
33 & a & b & c & d & e & f & & 68& a & b & c & d & e & f\\ \cline{1-7}\cline{9-15}
34 & a & b & c & d & e & f & & 69& a & b & c & d & e & f\\ \cline{1-7}\cline{9-15}
35 & a & b & c & d & e & f & & 70& a & b & c & d & e & f\\ \cline{1-7}\cline{9-15}
\end{tabular}\end{center}
\end{normalsize}
\newpage
\thispagestyle{empty}

\begin{center}
\begin{large}

King Fahd University of Petroleum and Minerals\\
Department of Mathematics and Statistics\\
\vspace*{0.5cm}
{\bf \fbox{CODE {\small{002}}}} \hfill {\bf Math 101} \hfill {\bf \fbox{CODE {\small{002}}}} \\
{\bf Final Exam}  \\
{\bf Term 162}  \\
{\bf Friday 26/5/2017}  \\
{\bf Net Time Allowed: 180 minutes}  \\
\vspace*{0.2cm}

\end{large}
\end{center}

\large{Name:  }\hrulefill

\vspace{3mm}

\large{ID: } \hrulefill \large{  Sec: } \hrulefill \large{.

\vspace{1cm}

\large{\bf{Check that this exam has {\underline{28}} questions.}}

\vspace{1cm}

\underline{{\large{\bf Important Instructions:}}}

\begin{enumerate}
\begin{normalsize}
\item  All types of calculators, pagers or mobile phones are NOT allowed during the examination.
\item  Use HB 2.5 pencils only.
\item  Use a good eraser. DO NOT use the erasers attached to the pencil.
\item  Write your name, ID number and Section number on the examination paper and in the upper left corner of the answer sheet.
\item  When bubbling your ID number and Section number, be sure that the bubbles match with the numbers that you write.
\item  The Test Code Number is already bubbled in your answer sheet. Make sure that it is the same as that printed on your question paper.
\item  When bubbling, make sure that the bubbled space is fully covered.
\item  When erasing a bubble, make sure that you do not leave any trace of penciling.
\end{normalsize}
\end{enumerate}

\newpage


\renewcommand{\thepage}{\noindent
Math 101, Final Exam, Term 162 \hfill Page {\bf \arabic{page} of 14} \hfill {\bf \fbox{002}}}

\setcounter{page}{1}

\begin{large}

\bn


\item %1
$\displaystyle \lim_{x\rightarrow 0}\, \left[ \frac{\sqrt{1+2x}-\sqrt{1-4x}}{x}\right]=$
\sc
\be
3
\ee
\be
1
\ee
\be
0
\ee
\be
2
\ee
\be
6
\ee
\v2



\item %2
If $\displaystyle \cosh\,x=\frac{5}{3}$ and $x>0,$ then
\sc
\be
\displaystyle \mbox{csch}\,x=\frac{4}{3}
\ee
\be
\displaystyle \sinh\,x=\frac{3}{4}
\ee
\be
\displaystyle \coth\,x=\frac{3}{5}
\ee
\be
\displaystyle \mbox{sech}\, x=1
\ee
\be
\displaystyle \tanh\,x=\frac{4}{5}
\ee

\newpage



\item %3
The sum of all positive real number $a$, that makes the function\\[0.2in] $\displaystyle f(x)= \left\{\begin{array}{lll} ax+3 & \mbox{if}\,x>a\\ \displaystyle x^2-x+2a^2 & \mbox{if}\,x\leq a  \\ \end{array}\right.$ continuous everywhere, is
\sc
\be
\displaystyle \frac{5}{2}
\ee
\be
2
\ee
\be
3
\ee
\be
\displaystyle 1
\ee
\be
\displaystyle \frac{3}{2}
\ee
\v2



\item %4
Let $\displaystyle h(x)=\frac{g(x)}{f(x)+g(x)}$. If $f(4)=1$, $g(4)=2$, $f'(4)=3$, and $g'(4)=-3$, then $h'(4)=$
\sc
\be
0
\ee
\be
-1
\ee
\be
-2
\ee
\be
-3
\ee
\be
3
\ee

\newpage



\item %5
$\displaystyle \tanh \, (\ln\,x)=$
\sc
\be
0
\ee
\be
\displaystyle \frac{x^2-1}{x^2+1}
\ee
\be
\displaystyle \frac{x^2+1}{x^2-1}
\ee
\be
\displaystyle \frac{\ln\,x^2-1}{\ln\,x^2+1}
\ee
\be
\displaystyle \frac{x-1}{x+1}
\ee
\v2



\item %6
Let $\displaystyle f(x)=x\sqrt{9-x^2}$ on the interval $[-3,3]$. The function $f$ attains its absolute maximum value at
\sc
\be
\displaystyle x=3
\ee
\be
\displaystyle x=\displaystyle-\frac{3}{\sqrt{2}}
\ee
\be
x=-3
\ee
\be
x=0
\ee
\be
\displaystyle x=\displaystyle\frac{3}{\sqrt{2}}
\ee

\newpage



\item %7
A particle moves in a straight line and has acceleration given by $a(t)=4t+10.$ Its initial velocity is $v(0)=-3$ feet/sec and its initial displacement $s(0)=5$ feet, its position is given by
\sc
\be
\displaystyle s(t)=3t^3+10t^2+3t-5
\ee
\be
\displaystyle s(t)=\frac{2}{3}\,t^3+5t^2-3t
\ee
\be
\displaystyle s(t)=\frac{2}{3}\,t^3+5t^2-3t+5
\ee
\be
\displaystyle s(t)=\frac{2}{3}\,t^5+5t^2+3t-5
\ee
\be
\displaystyle s(t)=3t^3+10t^2-3t+5
\ee
\v2



\item %8
Let $f(x)=c\,x+\ln(\cos x)$ where $c$ is a constant. The value of $c$ such that $f'(\displaystyle \frac{\pi}{4})=6$ equals to
\sc
\be
-2
\ee
\be
0
\ee
\be
6
\ee
\be
7
\ee
\be
1
\ee

\newpage



\item %9
Suppose $f''$ is continuous on $(-\infty,\infty).$ If $f'(2)=0$ and $f''(2)=-5$ then
\sc
\be
f\,\mbox{is increasing at}\,x=2
\ee
\be
f\,\mbox{is concave upward at}\,x=2
\ee
\be
f\,\mbox{has a point of inflection at}\,x=2
\ee
\be
f\,\mbox{has a local minimum at}\,x=2
\ee
\be
f\,\mbox{has a local maximum at}\,x=2
\ee
\v2



\item %10
The critical number(s) for the function $\displaystyle f(x)=x^2\,\ln\,x$ is(are)
\sc
\be
x=1
\ee
\be
\displaystyle x=\sqrt{e}
\ee
\be
\displaystyle x=\frac{1}{e}
\ee
\be
\displaystyle x=\displaystyle\frac{1}{\sqrt{e}}
\ee
\be
\displaystyle x=1\,\mbox{and}\,\sqrt{e}
\ee

\newpage



\item %11
The linearization  $L(x)$ of the function $f(x)=\sin\,x$ at \\$a=\pi/6$ is
\sc
\be
\displaystyle L(x)=x-\frac{\pi}{6}
\ee
\be
\displaystyle L(x)=\frac{\sqrt{3}}{2}+\frac{1}{2} \left(x-\frac{\pi}{6}\right)
\ee
\be
\displaystyle L(x)=\frac{1}{2}+\frac{\sqrt{3}}{2} \left(x-\frac{\pi}{6}\right)
\ee
\be
\displaystyle L(x)=1+x
\ee
\be
\displaystyle L(x)=1+\left(x-\frac{\pi}{6}\right)
\ee
\v2



\item %12
If $f(4)=-1$ and $f'(x)\geq 5$ for $2\leq x \leq 4,$ then the largest possible value of $f(2)$ is
\sc
\be
-1
\ee
\be
-11
\ee
\be
-5
\ee
\be
\displaystyle -\frac{1}{5}
\ee
\be
5
\ee

\newpage



\item %13
The graph of the derivative $f'$ of a \textbf{continuous} function $f$ is shown below, then which one of the following statements is \textbf{TRUE}?\\[0.5in]
\sc
\be
f\,\mbox{is increasing on the intervals}\,(0,2),(4,6),\,\mbox{and}\,(6,8)
\ee
\be
f\, \mbox{has an inflection point at}\,x=6
\ee
\be
f\, \mbox{has a local minimum at}\,x=4 \,\mbox{and}\, x=6
\ee
\be
f\, \mbox{is decreasing on the intervals}\, (0,2), (4,6),\,\mbox{and}\,(6,8)
\ee
\be
f\,\mbox{has a local maximum at}\,x=2 \,\mbox{and}\, x=6
\ee
\v2



\item %14
If $y\,\sec x=x\, \tan y \,+\, 1$, then $\displaystyle {\frac{dy}{dx}}$ when $x=0$ equals
\sc
\be
\displaystyle \frac{\sec\,1}{\tan\,1}
\ee
\be
\displaystyle \tan\,1
\ee
\be
1
\ee
\be
\displaystyle \sec\, 1
\ee
\be
\displaystyle \frac{1+\tan\,1}{\sec\,1}
\ee

\newpage



\item %15
$\displaystyle \lim_{x\rightarrow 1^+}\, \left[ \ln(x^8-1)-\ln(x^4-1)\right]=$
\sc
\be
\displaystyle \ln\, \left(\displaystyle\frac{1}{2}\right)
\ee
\be
\ln\,2
\ee
\be
\ln\,32
\ee
\be
0
\ee
\be
\displaystyle \ln\, \left(\displaystyle\frac{1}{32}\right)
\ee
\v2



\item %16
If $y=\sqrt{\sin \,x+\,y},$ then $\displaystyle \frac{dy}{dx}=$
\sc
\be
\displaystyle \frac{\cos\,x}{2y}
\ee
\be
\displaystyle \frac{\cos\,x}{1-2y}
\ee
\be
\displaystyle \frac{\cos\,x}{2y-1}
\ee
\be
\displaystyle \frac{\sin\, x}{2y-1}
\ee
\be
\displaystyle \frac{\cos\,x}{1-y}
\ee

\newpage



\item %17
If $\displaystyle y=\ln\,(1+\ln\,x)$ and $x>\,e$, then $y''=$
\sc
\be
\displaystyle \frac{-2-\ln\,x}{x^2(1+\ln\,x)^2}
\ee
\be
\displaystyle \frac{2+\ln\,x}{x^2(1+\ln\,x)^2}
\ee
\be
\displaystyle \frac{1}{1+\ln\,x}
\ee
\be
\displaystyle \frac{1}{x(1+\ln\,x)}
\ee
\be
\displaystyle \frac{-1}{x(1+\ln\,x)}
\ee
\v2



\item %18
The area of the largest rectangle that can be inscribed in a circle of radius $1$ is
\sc
\be
4
\ee
\be
2
\ee
\be
\displaystyle \frac{2}{\sqrt{2}}
\ee
\be
2\,\pi
\ee
\be
\displaystyle 2\,\sqrt{2}
\ee

\newpage



\item %19
The slope of the tangent line to the graph of $y=(x^2+1)^3\,e^{x^2}$ at $x=1$ equals
\sc
\be
20\,e
\ee
\be
5
\ee
\be
5\,e
\ee
\be
8\,e
\ee
\be
40\,e
\ee
\v2



\item %20
Which one of the following statements is \textbf{TRUE}?
\sc
\be
.
\ee
\be
.
\ee
\be
.
\ee
\be
.
\ee
\be
.
\ee

\newpage



\item %21
Which one of the following graphs represents the function $\displaystyle f(x)=\frac{x^{2}+1}{x}?$
\sc
\be
\mbox{graph}
\ee
\be
\mbox{graph}
\ee
\be
\mbox{graph}
\ee
\be
\mbox{graph}
\ee
\be
\mbox{graph}
\ee
\v2



\item %22
Let $A$ and $B$ be constants such that $y=A\,\sin\,x+B\,\cos\,x$ satisfies the equation $y''+y'-2y=\sin\,x$. Then
\sc
\be
\displaystyle A=\frac{-3}{10}\,\mbox{and}\,B=\frac{-1}{10}
\ee
\be
\displaystyle A=\frac{2}{5}\,\mbox{and}\,B=\frac{3}{5}
\ee
\be
\displaystyle A=\frac{-1}{3}\,\mbox{and}\,B=-1
\ee
\be
A=1\,\mbox{and}\, B=-1
\ee
\be
\displaystyle A=\frac{2}{3}\,\mbox{and}\,B=\frac{5}{2}
\ee

\newpage



\item %23
Suppose $f(x)= \left\{\begin{array}{lll} 2x & \mbox{for} & x\leq 0 \\ \displaystyle 2x-2x^2 & \mbox{for} & 0 < x <2\\ 2-6x & \mbox{for} & x\geq 2 \end{array}\right.$.\\ The function $f$ is \textbf{not differentiable} when
\sc
\be
x\in(0,2)
\ee
\be
x\in(-\infty,0]
\ee
\be
x=2
\ee
\be
x=0
\ee
\be
x=0\,\mbox{and}\,x=2
\ee
\v2



\item %24
One inflection point of the graph of $y=e^x\,\sin x$ on $[-\pi,\,\pi]$ is
\sc
\be
\displaystyle \left(\frac{\pi}{6},\, \frac{e^{\frac{\pi}{6}}}{2}\right)
\ee
\be
\displaystyle \left(\frac{\pi}{2},\, e^{\frac{\pi}{2}}\right)
\ee
\be
(\pi,0)
\ee
\be
\displaystyle \left(\frac{\pi}{3},\, \frac{\sqrt{3}\, e^{\frac{\pi}{2}}}{2}\right)
\ee
\be
(0,0)
\ee

\newpage



\item %25
Which one of the following statements is \textbf{FALSE}?
\sc
\be
.
\ee
\be
.
\ee
\be
.
\ee
\be
.
\ee
\be
.
\ee
\v2



\item %26
Let $f(x)=x^3-6x^2+9x+1$ and $x\in[2,4].$ If $M$ is the absolute maximum value and $m$ is the absolute minimum value, then $M+m=$
\sc
\be
1
\ee
\be
8
\ee
\be
6
\ee
\be
2
\ee
\be
4
\ee

\newpage



\item %27
The graph of $y=(1-x)\,e^x$ is concave downward on
\sc
\be
(0,1)
\ee
\be
(-1,0)
\ee
\be
(-\infty,1)
\ee
\be
(-e,-1)
\ee
\be
\displaystyle (-1,\infty)
\ee
\v2



\item %28
Using a linear approximation (or differentials), the best estimation to $\sqrt[3]{1001}$ is $10+B.$ Then $B=$
\sc
\be
\displaystyle \frac{5}{300}
\ee
\be
\displaystyle \frac{1}{100}
\ee
\be
\displaystyle \frac{1}{3}
\ee
\be
\displaystyle \frac{3}{10}
\ee
\be
\displaystyle \frac{1}{300}
\ee

\newpage



\en
\end{large}

\newpage


\renewcommand{\thepage}{\noindent Math 101, Final Exam, Term 162 \hfill Answer Sheet  \hfill {\bf \fbox{002}}}

\begin{Large}


\begin{tabular}{llll}
Name & .................................................& & \\
ID &   ................................& Sec & ..........\\
\end{tabular}

\vspace{10mm}


\end{Large}
\begin{normalsize}
\begin{center}
\begin{tabular}{|c|c c c c c c|c|c|c c c c c c|c|c|c c c c c c|}
\cline{1-7}\cline{9-15}
1  & a & b & c & d & e & f & \raisebox{0ex}[0cm][0cm]{\hspace{1cm}} & 36 & a & b & c & d & e & f\\ \cline{1-7}\cline{9-15}
2 & a & b & c & d & e & f & & 37& a & b & c & d & e & f\\ \cline{1-7}\cline{9-15}
3 & a & b & c & d & e & f & & 38& a & b & c & d & e & f\\ \cline{1-7}\cline{9-15}
4 & a & b & c & d & e & f & & 39& a & b & c & d & e & f\\ \cline{1-7}\cline{9-15}
5 & a & b & c & d & e & f & & 40& a & b & c & d & e & f\\ \cline{1-7}\cline{9-15}
6 & a & b & c & d & e & f & & 41& a & b & c & d & e & f\\ \cline{1-7}\cline{9-15}
7 & a & b & c & d & e & f & & 42& a & b & c & d & e & f\\ \cline{1-7}\cline{9-15}
8 & a & b & c & d & e & f & & 43& a & b & c & d & e & f\\ \cline{1-7}\cline{9-15}
9 & a & b & c & d & e & f & & 44& a & b & c & d & e & f\\ \cline{1-7}\cline{9-15}
10 & a & b & c & d & e & f & & 45& a & b & c & d & e & f\\ \cline{1-7}\cline{9-15}
11 & a & b & c & d & e & f & & 46& a & b & c & d & e & f\\ \cline{1-7}\cline{9-15}
12 & a & b & c & d & e & f & & 47& a & b & c & d & e & f\\ \cline{1-7}\cline{9-15}
13 & a & b & c & d & e & f & & 48& a & b & c & d & e & f\\ \cline{1-7}\cline{9-15}
14 & a & b & c & d & e & f & & 49& a & b & c & d & e & f\\ \cline{1-7}\cline{9-15}
15 & a & b & c & d & e & f & & 50& a & b & c & d & e & f\\ \cline{1-7}\cline{9-15}
16 & a & b & c & d & e & f & & 51& a & b & c & d & e & f\\ \cline{1-7}\cline{9-15}
17 & a & b & c & d & e & f & & 52& a & b & c & d & e & f\\ \cline{1-7}\cline{9-15}
18 & a & b & c & d & e & f & & 53& a & b & c & d & e & f\\ \cline{1-7}\cline{9-15}
19 & a & b & c & d & e & f & & 54& a & b & c & d & e & f\\ \cline{1-7}\cline{9-15}
20 & a & b & c & d & e & f & & 55& a & b & c & d & e & f\\ \cline{1-7}\cline{9-15}
21 & a & b & c & d & e & f & & 56& a & b & c & d & e & f\\ \cline{1-7}\cline{9-15}
22 & a & b & c & d & e & f & & 57& a & b & c & d & e & f\\ \cline{1-7}\cline{9-15}
23 & a & b & c & d & e & f & & 58& a & b & c & d & e & f\\ \cline{1-7}\cline{9-15}
24 & a & b & c & d & e & f & & 59& a & b & c & d & e & f\\ \cline{1-7}\cline{9-15}
25 & a & b & c & d & e & f & & 60& a & b & c & d & e & f\\ \cline{1-7}\cline{9-15}
26 & a & b & c & d & e & f & & 61& a & b & c & d & e & f\\ \cline{1-7}\cline{9-15}
27 & a & b & c & d & e & f & & 62& a & b & c & d & e & f\\ \cline{1-7}\cline{9-15}
28 & a & b & c & d & e & f & & 63& a & b & c & d & e & f\\ \cline{1-7}\cline{9-15}
29 & a & b & c & d & e & f & & 64& a & b & c & d & e & f\\ \cline{1-7}\cline{9-15}
30 & a & b & c & d & e & f & & 65& a & b & c & d & e & f\\ \cline{1-7}\cline{9-15}
31 & a & b & c & d & e & f & & 66& a & b & c & d & e & f\\ \cline{1-7}\cline{9-15}
32 & a & b & c & d & e & f & & 67& a & b & c & d & e & f\\ \cline{1-7}\cline{9-15}
33 & a & b & c & d & e & f & & 68& a & b & c & d & e & f\\ \cline{1-7}\cline{9-15}
34 & a & b & c & d & e & f & & 69& a & b & c & d & e & f\\ \cline{1-7}\cline{9-15}
35 & a & b & c & d & e & f & & 70& a & b & c & d & e & f\\ \cline{1-7}\cline{9-15}
\end{tabular}\end{center}
\end{normalsize}
\newpage
\thispagestyle{empty}

\begin{center}
\begin{large}

King Fahd University of Petroleum and Minerals\\
Department of Mathematics and Statistics\\
\vspace*{0.5cm}
{\bf \fbox{CODE {\small{003}}}} \hfill {\bf Math 101} \hfill {\bf \fbox{CODE {\small{003}}}} \\
{\bf Final Exam}  \\
{\bf Term 162}  \\
{\bf Friday 26/5/2017}  \\
{\bf Net Time Allowed: 180 minutes}  \\
\vspace*{0.2cm}

\end{large}
\end{center}

\large{Name:  }\hrulefill

\vspace{3mm}

\large{ID: } \hrulefill \large{  Sec: } \hrulefill \large{.

\vspace{1cm}

\large{\bf{Check that this exam has {\underline{28}} questions.}}

\vspace{1cm}

\underline{{\large{\bf Important Instructions:}}}

\begin{enumerate}
\begin{normalsize}
\item  All types of calculators, pagers or mobile phones are NOT allowed during the examination.
\item  Use HB 2.5 pencils only.
\item  Use a good eraser. DO NOT use the erasers attached to the pencil.
\item  Write your name, ID number and Section number on the examination paper and in the upper left corner of the answer sheet.
\item  When bubbling your ID number and Section number, be sure that the bubbles match with the numbers that you write.
\item  The Test Code Number is already bubbled in your answer sheet. Make sure that it is the same as that printed on your question paper.
\item  When bubbling, make sure that the bubbled space is fully covered.
\item  When erasing a bubble, make sure that you do not leave any trace of penciling.
\end{normalsize}
\end{enumerate}

\newpage


\renewcommand{\thepage}{\noindent
Math 101, Final Exam, Term 162 \hfill Page {\bf \arabic{page} of 14} \hfill {\bf \fbox{003}}}

\setcounter{page}{1}

\begin{large}

\bn


\item %1
The critical number(s) for the function $\displaystyle f(x)=x^2\,\ln\,x$ is(are)
\sc
\be
\displaystyle x=\sqrt{e}
\ee
\be
\displaystyle x=\frac{1}{e}
\ee
\be
\displaystyle x=1\,\mbox{and}\,\sqrt{e}
\ee
\be
\displaystyle x=\displaystyle\frac{1}{\sqrt{e}}
\ee
\be
x=1
\ee
\v2



\item %2
Suppose $f''$ is continuous on $(-\infty,\infty).$ If $f'(2)=0$ and $f''(2)=-5$ then
\sc
\be
f\,\mbox{is concave upward at}\,x=2
\ee
\be
f\,\mbox{is increasing at}\,x=2
\ee
\be
f\,\mbox{has a local maximum at}\,x=2
\ee
\be
f\,\mbox{has a point of inflection at}\,x=2
\ee
\be
f\,\mbox{has a local minimum at}\,x=2
\ee

\newpage



\item %3
$\displaystyle \lim_{x\rightarrow 0}\, \left[ \frac{\sqrt{1+2x}-\sqrt{1-4x}}{x}\right]=$
\sc
\be
0
\ee
\be
1
\ee
\be
3
\ee
\be
6
\ee
\be
2
\ee
\v2



\item %4
Let $\displaystyle h(x)=\frac{g(x)}{f(x)+g(x)}$. If $f(4)=1$, $g(4)=2$, $f'(4)=3$, and $g'(4)=-3$, then $h'(4)=$
\sc
\be
-3
\ee
\be
-1
\ee
\be
-2
\ee
\be
3
\ee
\be
0
\ee

\newpage



\item %5
Let $\displaystyle f(x)=x\sqrt{9-x^2}$ on the interval $[-3,3]$. The function $f$ attains its absolute maximum value at
\sc
\be
\displaystyle x=3
\ee
\be
x=-3
\ee
\be
\displaystyle x=\displaystyle\frac{3}{\sqrt{2}}
\ee
\be
\displaystyle x=\displaystyle-\frac{3}{\sqrt{2}}
\ee
\be
x=0
\ee
\v2



\item %6
A particle moves in a straight line and has acceleration given by $a(t)=4t+10.$ Its initial velocity is $v(0)=-3$ feet/sec and its initial displacement $s(0)=5$ feet, its position is given by
\sc
\be
\displaystyle s(t)=\frac{2}{3}\,t^5+5t^2+3t-5
\ee
\be
\displaystyle s(t)=3t^3+10t^2+3t-5
\ee
\be
\displaystyle s(t)=\frac{2}{3}\,t^3+5t^2-3t
\ee
\be
\displaystyle s(t)=3t^3+10t^2-3t+5
\ee
\be
\displaystyle s(t)=\frac{2}{3}\,t^3+5t^2-3t+5
\ee

\newpage



\item %7
The sum of all positive real number $a$, that makes the function\\[0.2in] $\displaystyle f(x)= \left\{\begin{array}{lll} ax+3 & \mbox{if}\,x>a\\ \displaystyle x^2-x+2a^2 & \mbox{if}\,x\leq a  \\ \end{array}\right.$ continuous everywhere, is
\sc
\be
\displaystyle 1
\ee
\be
3
\ee
\be
\displaystyle \frac{3}{2}
\ee
\be
2
\ee
\be
\displaystyle \frac{5}{2}
\ee
\v2



\item %8
$\displaystyle \tanh \, (\ln\,x)=$
\sc
\be
\displaystyle \frac{\ln\,x^2-1}{\ln\,x^2+1}
\ee
\be
\displaystyle \frac{x^2+1}{x^2-1}
\ee
\be
\displaystyle \frac{x-1}{x+1}
\ee
\be
0
\ee
\be
\displaystyle \frac{x^2-1}{x^2+1}
\ee

\newpage



\item %9
Let $f(x)=c\,x+\ln(\cos x)$ where $c$ is a constant. The value of $c$ such that $f'(\displaystyle \frac{\pi}{4})=6$ equals to
\sc
\be
6
\ee
\be
1
\ee
\be
0
\ee
\be
-2
\ee
\be
7
\ee
\v2



\item %10
If $\displaystyle \cosh\,x=\frac{5}{3}$ and $x>0,$ then
\sc
\be
\displaystyle \mbox{csch}\,x=\frac{4}{3}
\ee
\be
\displaystyle \tanh\,x=\frac{4}{5}
\ee
\be
\displaystyle \mbox{sech}\, x=1
\ee
\be
\displaystyle \coth\,x=\frac{3}{5}
\ee
\be
\displaystyle \sinh\,x=\frac{3}{4}
\ee

\newpage



\item %11
If $y\,\sec x=x\, \tan y \,+\, 1$, then $\displaystyle {\frac{dy}{dx}}$ when $x=0$ equals
\sc
\be
\displaystyle \frac{1+\tan\,1}{\sec\,1}
\ee
\be
\displaystyle \frac{\sec\,1}{\tan\,1}
\ee
\be
\displaystyle \tan\,1
\ee
\be
1
\ee
\be
\displaystyle \sec\, 1
\ee
\v2



\item %12
If $y=\sqrt{\sin \,x+\,y},$ then $\displaystyle \frac{dy}{dx}=$
\sc
\be
\displaystyle \frac{\sin\, x}{2y-1}
\ee
\be
\displaystyle \frac{\cos\,x}{2y}
\ee
\be
\displaystyle \frac{\cos\,x}{1-2y}
\ee
\be
\displaystyle \frac{\cos\,x}{2y-1}
\ee
\be
\displaystyle \frac{\cos\,x}{1-y}
\ee

\newpage



\item %13
If $\displaystyle y=\ln\,(1+\ln\,x)$ and $x>\,e$, then $y''=$
\sc
\be
\displaystyle \frac{-2-\ln\,x}{x^2(1+\ln\,x)^2}
\ee
\be
\displaystyle \frac{1}{x(1+\ln\,x)}
\ee
\be
\displaystyle \frac{2+\ln\,x}{x^2(1+\ln\,x)^2}
\ee
\be
\displaystyle \frac{1}{1+\ln\,x}
\ee
\be
\displaystyle \frac{-1}{x(1+\ln\,x)}
\ee
\v2



\item %14
Which one of the following statements is \textbf{TRUE}?
\sc
\be
.
\ee
\be
.
\ee
\be
.
\ee
\be
.
\ee
\be
.
\ee

\newpage



\item %15
The area of the largest rectangle that can be inscribed in a circle of radius $1$ is
\sc
\be
\displaystyle 2\,\sqrt{2}
\ee
\be
2
\ee
\be
\displaystyle \frac{2}{\sqrt{2}}
\ee
\be
2\,\pi
\ee
\be
4
\ee
\v2



\item %16
If $f(4)=-1$ and $f'(x)\geq 5$ for $2\leq x \leq 4,$ then the largest possible value of $f(2)$ is
\sc
\be
\displaystyle -\frac{1}{5}
\ee
\be
-1
\ee
\be
5
\ee
\be
-11
\ee
\be
-5
\ee

\newpage



\item %17
The graph of the derivative $f'$ of a \textbf{continuous} function $f$ is shown below, then which one of the following statements is \textbf{TRUE}?\\[0.5in]
\sc
\be
f\, \mbox{has an inflection point at}\,x=6
\ee
\be
f\, \mbox{has a local minimum at}\,x=4 \,\mbox{and}\, x=6
\ee
\be
f\,\mbox{is increasing on the intervals}\,(0,2),(4,6),\,\mbox{and}\,(6,8)
\ee
\be
f\, \mbox{is decreasing on the intervals}\, (0,2), (4,6),\,\mbox{and}\,(6,8)
\ee
\be
f\,\mbox{has a local maximum at}\,x=2 \,\mbox{and}\, x=6
\ee
\v2



\item %18
The slope of the tangent line to the graph of $y=(x^2+1)^3\,e^{x^2}$ at $x=1$ equals
\sc
\be
40\,e
\ee
\be
5\,e
\ee
\be
5
\ee
\be
20\,e
\ee
\be
8\,e
\ee

\newpage



\item %19
The linearization  $L(x)$ of the function $f(x)=\sin\,x$ at \\$a=\pi/6$ is
\sc
\be
\displaystyle L(x)=1+\left(x-\frac{\pi}{6}\right)
\ee
\be
\displaystyle L(x)=1+x
\ee
\be
\displaystyle L(x)=\frac{1}{2}+\frac{\sqrt{3}}{2} \left(x-\frac{\pi}{6}\right)
\ee
\be
\displaystyle L(x)=\frac{\sqrt{3}}{2}+\frac{1}{2} \left(x-\frac{\pi}{6}\right)
\ee
\be
\displaystyle L(x)=x-\frac{\pi}{6}
\ee
\v2



\item %20
$\displaystyle \lim_{x\rightarrow 1^+}\, \left[ \ln(x^8-1)-\ln(x^4-1)\right]=$
\sc
\be
\displaystyle \ln\, \left(\displaystyle\frac{1}{2}\right)
\ee
\be
0
\ee
\be
\displaystyle \ln\, \left(\displaystyle\frac{1}{32}\right)
\ee
\be
\ln\,32
\ee
\be
\ln\,2
\ee

\newpage



\item %21
Using a linear approximation (or differentials), the best estimation to $\sqrt[3]{1001}$ is $10+B.$ Then $B=$
\sc
\be
\displaystyle \frac{1}{300}
\ee
\be
\displaystyle \frac{1}{100}
\ee
\be
\displaystyle \frac{3}{10}
\ee
\be
\displaystyle \frac{5}{300}
\ee
\be
\displaystyle \frac{1}{3}
\ee
\v2



\item %22
Which one of the following statements is \textbf{FALSE}?
\sc
\be
.
\ee
\be
.
\ee
\be
.
\ee
\be
.
\ee
\be
.
\ee

\newpage



\item %23
Which one of the following graphs represents the function $\displaystyle f(x)=\frac{x^{2}+1}{x}?$
\sc
\be
\mbox{graph}
\ee
\be
\mbox{graph}
\ee
\be
\mbox{graph}
\ee
\be
\mbox{graph}
\ee
\be
\mbox{graph}
\ee
\v2



\item %24
One inflection point of the graph of $y=e^x\,\sin x$ on $[-\pi,\,\pi]$ is
\sc
\be
\displaystyle \left(\frac{\pi}{2},\, e^{\frac{\pi}{2}}\right)
\ee
\be
\displaystyle \left(\frac{\pi}{3},\, \frac{\sqrt{3}\, e^{\frac{\pi}{2}}}{2}\right)
\ee
\be
(\pi,0)
\ee
\be
(0,0)
\ee
\be
\displaystyle \left(\frac{\pi}{6},\, \frac{e^{\frac{\pi}{6}}}{2}\right)
\ee

\newpage



\item %25
Let $A$ and $B$ be constants such that $y=A\,\sin\,x+B\,\cos\,x$ satisfies the equation $y''+y'-2y=\sin\,x$. Then
\sc
\be
\displaystyle A=\frac{2}{5}\,\mbox{and}\,B=\frac{3}{5}
\ee
\be
\displaystyle A=\frac{-1}{3}\,\mbox{and}\,B=-1
\ee
\be
A=1\,\mbox{and}\, B=-1
\ee
\be
\displaystyle A=\frac{2}{3}\,\mbox{and}\,B=\frac{5}{2}
\ee
\be
\displaystyle A=\frac{-3}{10}\,\mbox{and}\,B=\frac{-1}{10}
\ee
\v2



\item %26
Let $f(x)=x^3-6x^2+9x+1$ and $x\in[2,4].$ If $M$ is the absolute maximum value and $m$ is the absolute minimum value, then $M+m=$
\sc
\be
4
\ee
\be
8
\ee
\be
1
\ee
\be
2
\ee
\be
6
\ee

\newpage



\item %27
Suppose $f(x)= \left\{\begin{array}{lll} 2x & \mbox{for} & x\leq 0 \\ \displaystyle 2x-2x^2 & \mbox{for} & 0 < x <2\\ 2-6x & \mbox{for} & x\geq 2 \end{array}\right.$.\\ The function $f$ is \textbf{not differentiable} when
\sc
\be
x\in(-\infty,0]
\ee
\be
x\in(0,2)
\ee
\be
x=2
\ee
\be
x=0
\ee
\be
x=0\,\mbox{and}\,x=2
\ee
\v2



\item %28
The graph of $y=(1-x)\,e^x$ is concave downward on
\sc
\be
\displaystyle (-1,\infty)
\ee
\be
(-\infty,1)
\ee
\be
(-e,-1)
\ee
\be
(0,1)
\ee
\be
(-1,0)
\ee

\newpage



\en
\end{large}

\newpage


\renewcommand{\thepage}{\noindent Math 101, Final Exam, Term 162 \hfill Answer Sheet  \hfill {\bf \fbox{003}}}

\begin{Large}


\begin{tabular}{llll}
Name & .................................................& & \\
ID &   ................................& Sec & ..........\\
\end{tabular}

\vspace{10mm}


\end{Large}
\begin{normalsize}
\begin{center}
\begin{tabular}{|c|c c c c c c|c|c|c c c c c c|c|c|c c c c c c|}
\cline{1-7}\cline{9-15}
1  & a & b & c & d & e & f & \raisebox{0ex}[0cm][0cm]{\hspace{1cm}} & 36 & a & b & c & d & e & f\\ \cline{1-7}\cline{9-15}
2 & a & b & c & d & e & f & & 37& a & b & c & d & e & f\\ \cline{1-7}\cline{9-15}
3 & a & b & c & d & e & f & & 38& a & b & c & d & e & f\\ \cline{1-7}\cline{9-15}
4 & a & b & c & d & e & f & & 39& a & b & c & d & e & f\\ \cline{1-7}\cline{9-15}
5 & a & b & c & d & e & f & & 40& a & b & c & d & e & f\\ \cline{1-7}\cline{9-15}
6 & a & b & c & d & e & f & & 41& a & b & c & d & e & f\\ \cline{1-7}\cline{9-15}
7 & a & b & c & d & e & f & & 42& a & b & c & d & e & f\\ \cline{1-7}\cline{9-15}
8 & a & b & c & d & e & f & & 43& a & b & c & d & e & f\\ \cline{1-7}\cline{9-15}
9 & a & b & c & d & e & f & & 44& a & b & c & d & e & f\\ \cline{1-7}\cline{9-15}
10 & a & b & c & d & e & f & & 45& a & b & c & d & e & f\\ \cline{1-7}\cline{9-15}
11 & a & b & c & d & e & f & & 46& a & b & c & d & e & f\\ \cline{1-7}\cline{9-15}
12 & a & b & c & d & e & f & & 47& a & b & c & d & e & f\\ \cline{1-7}\cline{9-15}
13 & a & b & c & d & e & f & & 48& a & b & c & d & e & f\\ \cline{1-7}\cline{9-15}
14 & a & b & c & d & e & f & & 49& a & b & c & d & e & f\\ \cline{1-7}\cline{9-15}
15 & a & b & c & d & e & f & & 50& a & b & c & d & e & f\\ \cline{1-7}\cline{9-15}
16 & a & b & c & d & e & f & & 51& a & b & c & d & e & f\\ \cline{1-7}\cline{9-15}
17 & a & b & c & d & e & f & & 52& a & b & c & d & e & f\\ \cline{1-7}\cline{9-15}
18 & a & b & c & d & e & f & & 53& a & b & c & d & e & f\\ \cline{1-7}\cline{9-15}
19 & a & b & c & d & e & f & & 54& a & b & c & d & e & f\\ \cline{1-7}\cline{9-15}
20 & a & b & c & d & e & f & & 55& a & b & c & d & e & f\\ \cline{1-7}\cline{9-15}
21 & a & b & c & d & e & f & & 56& a & b & c & d & e & f\\ \cline{1-7}\cline{9-15}
22 & a & b & c & d & e & f & & 57& a & b & c & d & e & f\\ \cline{1-7}\cline{9-15}
23 & a & b & c & d & e & f & & 58& a & b & c & d & e & f\\ \cline{1-7}\cline{9-15}
24 & a & b & c & d & e & f & & 59& a & b & c & d & e & f\\ \cline{1-7}\cline{9-15}
25 & a & b & c & d & e & f & & 60& a & b & c & d & e & f\\ \cline{1-7}\cline{9-15}
26 & a & b & c & d & e & f & & 61& a & b & c & d & e & f\\ \cline{1-7}\cline{9-15}
27 & a & b & c & d & e & f & & 62& a & b & c & d & e & f\\ \cline{1-7}\cline{9-15}
28 & a & b & c & d & e & f & & 63& a & b & c & d & e & f\\ \cline{1-7}\cline{9-15}
29 & a & b & c & d & e & f & & 64& a & b & c & d & e & f\\ \cline{1-7}\cline{9-15}
30 & a & b & c & d & e & f & & 65& a & b & c & d & e & f\\ \cline{1-7}\cline{9-15}
31 & a & b & c & d & e & f & & 66& a & b & c & d & e & f\\ \cline{1-7}\cline{9-15}
32 & a & b & c & d & e & f & & 67& a & b & c & d & e & f\\ \cline{1-7}\cline{9-15}
33 & a & b & c & d & e & f & & 68& a & b & c & d & e & f\\ \cline{1-7}\cline{9-15}
34 & a & b & c & d & e & f & & 69& a & b & c & d & e & f\\ \cline{1-7}\cline{9-15}
35 & a & b & c & d & e & f & & 70& a & b & c & d & e & f\\ \cline{1-7}\cline{9-15}
\end{tabular}\end{center}
\end{normalsize}
\newpage
\thispagestyle{empty}

\begin{center}
\begin{large}

King Fahd University of Petroleum and Minerals\\
Department of Mathematics and Statistics\\
\vspace*{0.5cm}
{\bf \fbox{CODE {\small{004}}}} \hfill {\bf Math 101} \hfill {\bf \fbox{CODE {\small{004}}}} \\
{\bf Final Exam}  \\
{\bf Term 162}  \\
{\bf Friday 26/5/2017}  \\
{\bf Net Time Allowed: 180 minutes}  \\
\vspace*{0.2cm}

\end{large}
\end{center}

\large{Name:  }\hrulefill

\vspace{3mm}

\large{ID: } \hrulefill \large{  Sec: } \hrulefill \large{.

\vspace{1cm}

\large{\bf{Check that this exam has {\underline{28}} questions.}}

\vspace{1cm}

\underline{{\large{\bf Important Instructions:}}}

\begin{enumerate}
\begin{normalsize}
\item  All types of calculators, pagers or mobile phones are NOT allowed during the examination.
\item  Use HB 2.5 pencils only.
\item  Use a good eraser. DO NOT use the erasers attached to the pencil.
\item  Write your name, ID number and Section number on the examination paper and in the upper left corner of the answer sheet.
\item  When bubbling your ID number and Section number, be sure that the bubbles match with the numbers that you write.
\item  The Test Code Number is already bubbled in your answer sheet. Make sure that it is the same as that printed on your question paper.
\item  When bubbling, make sure that the bubbled space is fully covered.
\item  When erasing a bubble, make sure that you do not leave any trace of penciling.
\end{normalsize}
\end{enumerate}

\newpage


\renewcommand{\thepage}{\noindent
Math 101, Final Exam, Term 162 \hfill Page {\bf \arabic{page} of 14} \hfill {\bf \fbox{004}}}

\setcounter{page}{1}

\begin{large}

\bn


\item %1
A particle moves in a straight line and has acceleration given by $a(t)=4t+10.$ Its initial velocity is $v(0)=-3$ feet/sec and its initial displacement $s(0)=5$ feet, its position is given by
\sc
\be
\displaystyle s(t)=\frac{2}{3}\,t^3+5t^2-3t
\ee
\be
\displaystyle s(t)=\frac{2}{3}\,t^3+5t^2-3t+5
\ee
\be
\displaystyle s(t)=3t^3+10t^2+3t-5
\ee
\be
\displaystyle s(t)=3t^3+10t^2-3t+5
\ee
\be
\displaystyle s(t)=\frac{2}{3}\,t^5+5t^2+3t-5
\ee
\v2



\item %2
$\displaystyle \lim_{x\rightarrow 0}\, \left[ \frac{\sqrt{1+2x}-\sqrt{1-4x}}{x}\right]=$
\sc
\be
2
\ee
\be
3
\ee
\be
0
\ee
\be
1
\ee
\be
6
\ee

\newpage



\item %3
Let $f(x)=c\,x+\ln(\cos x)$ where $c$ is a constant. The value of $c$ such that $f'(\displaystyle \frac{\pi}{4})=6$ equals to
\sc
\be
6
\ee
\be
0
\ee
\be
1
\ee
\be
-2
\ee
\be
7
\ee
\v2



\item %4
Let $\displaystyle f(x)=x\sqrt{9-x^2}$ on the interval $[-3,3]$. The function $f$ attains its absolute maximum value at
\sc
\be
\displaystyle x=\displaystyle\frac{3}{\sqrt{2}}
\ee
\be
\displaystyle x=3
\ee
\be
\displaystyle x=\displaystyle-\frac{3}{\sqrt{2}}
\ee
\be
x=-3
\ee
\be
x=0
\ee

\newpage



\item %5
Let $\displaystyle h(x)=\frac{g(x)}{f(x)+g(x)}$. If $f(4)=1$, $g(4)=2$, $f'(4)=3$, and $g'(4)=-3$, then $h'(4)=$
\sc
\be
-2
\ee
\be
0
\ee
\be
-1
\ee
\be
-3
\ee
\be
3
\ee
\v2



\item %6
The sum of all positive real number $a$, that makes the function\\[0.2in] $\displaystyle f(x)= \left\{\begin{array}{lll} ax+3 & \mbox{if}\,x>a\\ \displaystyle x^2-x+2a^2 & \mbox{if}\,x\leq a  \\ \end{array}\right.$ continuous everywhere, is
\sc
\be
2
\ee
\be
\displaystyle 1
\ee
\be
3
\ee
\be
\displaystyle \frac{5}{2}
\ee
\be
\displaystyle \frac{3}{2}
\ee

\newpage



\item %7
The critical number(s) for the function $\displaystyle f(x)=x^2\,\ln\,x$ is(are)
\sc
\be
\displaystyle x=\sqrt{e}
\ee
\be
\displaystyle x=\displaystyle\frac{1}{\sqrt{e}}
\ee
\be
\displaystyle x=\frac{1}{e}
\ee
\be
x=1
\ee
\be
\displaystyle x=1\,\mbox{and}\,\sqrt{e}
\ee
\v2



\item %8
$\displaystyle \tanh \, (\ln\,x)=$
\sc
\be
\displaystyle \frac{x^2+1}{x^2-1}
\ee
\be
0
\ee
\be
\displaystyle \frac{x-1}{x+1}
\ee
\be
\displaystyle \frac{x^2-1}{x^2+1}
\ee
\be
\displaystyle \frac{\ln\,x^2-1}{\ln\,x^2+1}
\ee

\newpage



\item %9
If $\displaystyle \cosh\,x=\frac{5}{3}$ and $x>0,$ then
\sc
\be
\displaystyle \sinh\,x=\frac{3}{4}
\ee
\be
\displaystyle \mbox{sech}\, x=1
\ee
\be
\displaystyle \coth\,x=\frac{3}{5}
\ee
\be
\displaystyle \tanh\,x=\frac{4}{5}
\ee
\be
\displaystyle \mbox{csch}\,x=\frac{4}{3}
\ee
\v2



\item %10
Suppose $f''$ is continuous on $(-\infty,\infty).$ If $f'(2)=0$ and $f''(2)=-5$ then
\sc
\be
f\,\mbox{has a local maximum at}\,x=2
\ee
\be
f\,\mbox{is concave upward at}\,x=2
\ee
\be
f\,\mbox{has a local minimum at}\,x=2
\ee
\be
f\,\mbox{has a point of inflection at}\,x=2
\ee
\be
f\,\mbox{is increasing at}\,x=2
\ee

\newpage



\item %11
$\displaystyle \lim_{x\rightarrow 1^+}\, \left[ \ln(x^8-1)-\ln(x^4-1)\right]=$
\sc
\be
\ln\,32
\ee
\be
\displaystyle \ln\, \left(\displaystyle\frac{1}{2}\right)
\ee
\be
0
\ee
\be
\ln\,2
\ee
\be
\displaystyle \ln\, \left(\displaystyle\frac{1}{32}\right)
\ee
\v2



\item %12
Which one of the following statements is \textbf{TRUE}?
\sc
\be
.
\ee
\be
.
\ee
\be
.
\ee
\be
.
\ee
\be
.
\ee

\newpage



\item %13
If $y\,\sec x=x\, \tan y \,+\, 1$, then $\displaystyle {\frac{dy}{dx}}$ when $x=0$ equals
\sc
\be
\displaystyle \frac{1+\tan\,1}{\sec\,1}
\ee
\be
\displaystyle \sec\, 1
\ee
\be
\displaystyle \frac{\sec\,1}{\tan\,1}
\ee
\be
1
\ee
\be
\displaystyle \tan\,1
\ee
\v2



\item %14
The linearization  $L(x)$ of the function $f(x)=\sin\,x$ at \\$a=\pi/6$ is
\sc
\be
\displaystyle L(x)=1+x
\ee
\be
\displaystyle L(x)=1+\left(x-\frac{\pi}{6}\right)
\ee
\be
\displaystyle L(x)=x-\frac{\pi}{6}
\ee
\be
\displaystyle L(x)=\frac{1}{2}+\frac{\sqrt{3}}{2} \left(x-\frac{\pi}{6}\right)
\ee
\be
\displaystyle L(x)=\frac{\sqrt{3}}{2}+\frac{1}{2} \left(x-\frac{\pi}{6}\right)
\ee

\newpage



\item %15
The graph of the derivative $f'$ of a \textbf{continuous} function $f$ is shown below, then which one of the following statements is \textbf{TRUE}?\\[0.5in]
\sc
\be
f\, \mbox{is decreasing on the intervals}\, (0,2), (4,6),\,\mbox{and}\,(6,8)
\ee
\be
f\,\mbox{has a local maximum at}\,x=2 \,\mbox{and}\, x=6
\ee
\be
f\,\mbox{is increasing on the intervals}\,(0,2),(4,6),\,\mbox{and}\,(6,8)
\ee
\be
f\, \mbox{has a local minimum at}\,x=4 \,\mbox{and}\, x=6
\ee
\be
f\, \mbox{has an inflection point at}\,x=6
\ee
\v2



\item %16
The slope of the tangent line to the graph of $y=(x^2+1)^3\,e^{x^2}$ at $x=1$ equals
\sc
\be
5\,e
\ee
\be
40\,e
\ee
\be
20\,e
\ee
\be
8\,e
\ee
\be
5
\ee

\newpage



\item %17
If $y=\sqrt{\sin \,x+\,y},$ then $\displaystyle \frac{dy}{dx}=$
\sc
\be
\displaystyle \frac{\cos\,x}{1-y}
\ee
\be
\displaystyle \frac{\cos\,x}{2y}
\ee
\be
\displaystyle \frac{\cos\,x}{2y-1}
\ee
\be
\displaystyle \frac{\sin\, x}{2y-1}
\ee
\be
\displaystyle \frac{\cos\,x}{1-2y}
\ee
\v2



\item %18
If $\displaystyle y=\ln\,(1+\ln\,x)$ and $x>\,e$, then $y''=$
\sc
\be
\displaystyle \frac{1}{1+\ln\,x}
\ee
\be
\displaystyle \frac{-2-\ln\,x}{x^2(1+\ln\,x)^2}
\ee
\be
\displaystyle \frac{1}{x(1+\ln\,x)}
\ee
\be
\displaystyle \frac{2+\ln\,x}{x^2(1+\ln\,x)^2}
\ee
\be
\displaystyle \frac{-1}{x(1+\ln\,x)}
\ee

\newpage



\item %19
The area of the largest rectangle that can be inscribed in a circle of radius $1$ is
\sc
\be
\displaystyle \frac{2}{\sqrt{2}}
\ee
\be
4
\ee
\be
2\,\pi
\ee
\be
\displaystyle 2\,\sqrt{2}
\ee
\be
2
\ee
\v2



\item %20
If $f(4)=-1$ and $f'(x)\geq 5$ for $2\leq x \leq 4,$ then the largest possible value of $f(2)$ is
\sc
\be
-5
\ee
\be
-11
\ee
\be
-1
\ee
\be
5
\ee
\be
\displaystyle -\frac{1}{5}
\ee

\newpage



\item %21
Using a linear approximation (or differentials), the best estimation to $\sqrt[3]{1001}$ is $10+B.$ Then $B=$
\sc
\be
\displaystyle \frac{1}{100}
\ee
\be
\displaystyle \frac{5}{300}
\ee
\be
\displaystyle \frac{1}{3}
\ee
\be
\displaystyle \frac{3}{10}
\ee
\be
\displaystyle \frac{1}{300}
\ee
\v2



\item %22
Which one of the following statements is \textbf{FALSE}?
\sc
\be
.
\ee
\be
.
\ee
\be
.
\ee
\be
.
\ee
\be
.
\ee

\newpage



\item %23
Suppose $f(x)= \left\{\begin{array}{lll} 2x & \mbox{for} & x\leq 0 \\ \displaystyle 2x-2x^2 & \mbox{for} & 0 < x <2\\ 2-6x & \mbox{for} & x\geq 2 \end{array}\right.$.\\ The function $f$ is \textbf{not differentiable} when
\sc
\be
x\in(0,2)
\ee
\be
x=0
\ee
\be
x=2
\ee
\be
x=0\,\mbox{and}\,x=2
\ee
\be
x\in(-\infty,0]
\ee
\v2



\item %24
One inflection point of the graph of $y=e^x\,\sin x$ on $[-\pi,\,\pi]$ is
\sc
\be
\displaystyle \left(\frac{\pi}{2},\, e^{\frac{\pi}{2}}\right)
\ee
\be
\displaystyle \left(\frac{\pi}{6},\, \frac{e^{\frac{\pi}{6}}}{2}\right)
\ee
\be
(0,0)
\ee
\be
\displaystyle \left(\frac{\pi}{3},\, \frac{\sqrt{3}\, e^{\frac{\pi}{2}}}{2}\right)
\ee
\be
(\pi,0)
\ee

\newpage



\item %25
Let $f(x)=x^3-6x^2+9x+1$ and $x\in[2,4].$ If $M$ is the absolute maximum value and $m$ is the absolute minimum value, then $M+m=$
\sc
\be
8
\ee
\be
4
\ee
\be
2
\ee
\be
6
\ee
\be
1
\ee
\v2



\item %26
Which one of the following graphs represents the function $\displaystyle f(x)=\frac{x^{2}+1}{x}?$
\sc
\be
\mbox{graph}
\ee
\be
\mbox{graph}
\ee
\be
\mbox{graph}
\ee
\be
\mbox{graph}
\ee
\be
\mbox{graph}
\ee

\newpage



\item %27
Let $A$ and $B$ be constants such that $y=A\,\sin\,x+B\,\cos\,x$ satisfies the equation $y''+y'-2y=\sin\,x$. Then
\sc
\be
\displaystyle A=\frac{-1}{3}\,\mbox{and}\,B=-1
\ee
\be
\displaystyle A=\frac{2}{3}\,\mbox{and}\,B=\frac{5}{2}
\ee
\be
\displaystyle A=\frac{-3}{10}\,\mbox{and}\,B=\frac{-1}{10}
\ee
\be
A=1\,\mbox{and}\, B=-1
\ee
\be
\displaystyle A=\frac{2}{5}\,\mbox{and}\,B=\frac{3}{5}
\ee
\v2



\item %28
The graph of $y=(1-x)\,e^x$ is concave downward on
\sc
\be
\displaystyle (-1,\infty)
\ee
\be
(-\infty,1)
\ee
\be
(-e,-1)
\ee
\be
(-1,0)
\ee
\be
(0,1)
\ee

\newpage



\en
\end{large}

\newpage


\renewcommand{\thepage}{\noindent Math 101, Final Exam, Term 162 \hfill Answer Sheet  \hfill {\bf \fbox{004}}}

\begin{Large}


\begin{tabular}{llll}
Name & .................................................& & \\
ID &   ................................& Sec & ..........\\
\end{tabular}

\vspace{10mm}


\end{Large}
\begin{normalsize}
\begin{center}
\begin{tabular}{|c|c c c c c c|c|c|c c c c c c|c|c|c c c c c c|}
\cline{1-7}\cline{9-15}
1  & a & b & c & d & e & f & \raisebox{0ex}[0cm][0cm]{\hspace{1cm}} & 36 & a & b & c & d & e & f\\ \cline{1-7}\cline{9-15}
2 & a & b & c & d & e & f & & 37& a & b & c & d & e & f\\ \cline{1-7}\cline{9-15}
3 & a & b & c & d & e & f & & 38& a & b & c & d & e & f\\ \cline{1-7}\cline{9-15}
4 & a & b & c & d & e & f & & 39& a & b & c & d & e & f\\ \cline{1-7}\cline{9-15}
5 & a & b & c & d & e & f & & 40& a & b & c & d & e & f\\ \cline{1-7}\cline{9-15}
6 & a & b & c & d & e & f & & 41& a & b & c & d & e & f\\ \cline{1-7}\cline{9-15}
7 & a & b & c & d & e & f & & 42& a & b & c & d & e & f\\ \cline{1-7}\cline{9-15}
8 & a & b & c & d & e & f & & 43& a & b & c & d & e & f\\ \cline{1-7}\cline{9-15}
9 & a & b & c & d & e & f & & 44& a & b & c & d & e & f\\ \cline{1-7}\cline{9-15}
10 & a & b & c & d & e & f & & 45& a & b & c & d & e & f\\ \cline{1-7}\cline{9-15}
11 & a & b & c & d & e & f & & 46& a & b & c & d & e & f\\ \cline{1-7}\cline{9-15}
12 & a & b & c & d & e & f & & 47& a & b & c & d & e & f\\ \cline{1-7}\cline{9-15}
13 & a & b & c & d & e & f & & 48& a & b & c & d & e & f\\ \cline{1-7}\cline{9-15}
14 & a & b & c & d & e & f & & 49& a & b & c & d & e & f\\ \cline{1-7}\cline{9-15}
15 & a & b & c & d & e & f & & 50& a & b & c & d & e & f\\ \cline{1-7}\cline{9-15}
16 & a & b & c & d & e & f & & 51& a & b & c & d & e & f\\ \cline{1-7}\cline{9-15}
17 & a & b & c & d & e & f & & 52& a & b & c & d & e & f\\ \cline{1-7}\cline{9-15}
18 & a & b & c & d & e & f & & 53& a & b & c & d & e & f\\ \cline{1-7}\cline{9-15}
19 & a & b & c & d & e & f & & 54& a & b & c & d & e & f\\ \cline{1-7}\cline{9-15}
20 & a & b & c & d & e & f & & 55& a & b & c & d & e & f\\ \cline{1-7}\cline{9-15}
21 & a & b & c & d & e & f & & 56& a & b & c & d & e & f\\ \cline{1-7}\cline{9-15}
22 & a & b & c & d & e & f & & 57& a & b & c & d & e & f\\ \cline{1-7}\cline{9-15}
23 & a & b & c & d & e & f & & 58& a & b & c & d & e & f\\ \cline{1-7}\cline{9-15}
24 & a & b & c & d & e & f & & 59& a & b & c & d & e & f\\ \cline{1-7}\cline{9-15}
25 & a & b & c & d & e & f & & 60& a & b & c & d & e & f\\ \cline{1-7}\cline{9-15}
26 & a & b & c & d & e & f & & 61& a & b & c & d & e & f\\ \cline{1-7}\cline{9-15}
27 & a & b & c & d & e & f & & 62& a & b & c & d & e & f\\ \cline{1-7}\cline{9-15}
28 & a & b & c & d & e & f & & 63& a & b & c & d & e & f\\ \cline{1-7}\cline{9-15}
29 & a & b & c & d & e & f & & 64& a & b & c & d & e & f\\ \cline{1-7}\cline{9-15}
30 & a & b & c & d & e & f & & 65& a & b & c & d & e & f\\ \cline{1-7}\cline{9-15}
31 & a & b & c & d & e & f & & 66& a & b & c & d & e & f\\ \cline{1-7}\cline{9-15}
32 & a & b & c & d & e & f & & 67& a & b & c & d & e & f\\ \cline{1-7}\cline{9-15}
33 & a & b & c & d & e & f & & 68& a & b & c & d & e & f\\ \cline{1-7}\cline{9-15}
34 & a & b & c & d & e & f & & 69& a & b & c & d & e & f\\ \cline{1-7}\cline{9-15}
35 & a & b & c & d & e & f & & 70& a & b & c & d & e & f\\ \cline{1-7}\cline{9-15}
\end{tabular}\end{center}
\end{normalsize}
\newpage
\renewcommand{\thepage}{\noindent Math 101, Final Exam, Term 162 \hfill \arabic{page} \hfill {\bf \fbox{ANSWER KEY}}}
\begin{normalsize}
\setcounter{page}{1}
\vspace {1cm}


\begin{center}
\begin{tabular}{|c||c|c|c|c|c|}
\hline
Q & MM & V1& V2& V3& V4\\
\hline \hline
1 & a  & a & a & d & b \\ \hline
2 & a  & d & e & c & b \\ \hline
3 & a  & a & e & c & e \\ \hline
4 & a  & d & b & b & a \\ \hline
5 & a  & c & b & c & c \\ \hline
6 & a  & d & e & e & e \\ \hline
7 & a  & a & c & c & b \\ \hline
8 & a  & e & d & e & d \\ \hline
9 & a  & d & e & e & d \\ \hline
10 & a  & a & d & b & a \\ \hline
11 & a  & b & c & c & d \\ \hline
12 & a  & b & b & d & a \\ \hline
13 & a  & b & e & a & e \\ \hline
14 & a  & d & b & c & d \\ \hline
15 & a  & a & b & b & b \\ \hline
16 & a  & c & c & d & b \\ \hline
17 & a  & e & a & e & c \\ \hline
18 & a  & e & b & a & b \\ \hline
19 & a  & a & e & c & e \\ \hline
20 & a  & d & d & e & b \\ \hline
21 & a  & c & c & a & e \\ \hline
22 & a  & a & a & a & b \\ \hline
23 & a  & b & c & b & c \\ \hline
24 & a  & b & b & a & a \\ \hline
25 & a  & d & c & e & d \\ \hline
26 & a  & a & c & e & e \\ \hline
27 & a  & b & e & c & c \\ \hline
28 & a  & b & e & a & a \\ \hline
\end{tabular}
\end{center}
\newpage
\end{normalsize}
\renewcommand{\thepage}{\noindent Math 101, Final Exam, Term 162 \hfill \arabic{page} \hfill {\bf \fbox{Answer Counts}}}

\begin{normalsize}
\begin{center}
\vspace {1cm}

\begin{Large}
Answer Counts \\
\end{Large}
\vspace {1cm}
\begin{tabular}{|c||c|c|c|c|c|c|}
\hline
V & a & b & c & d & e \\ \hline \hline
1 & 7 & 5 & 6 & 5 & 5 \\ \hline
2 & 7 & 5 & 6 & 4 & 6 \\ \hline
3 & 4 & 5 & 6 & 9 & 4 \\ \hline
4 & 8 & 7 & 7 & 1 & 5 \\ \hline
\end{tabular}
\end{center}
\end{normalsize}
\newpage
\end{document}
